Wie du dem folgenden Stundenplan entnehmen kannst, enthält das Lehramtsstudium der Informatik im ersten Semester zunächst nur die Informatik II-Vorlesung . %Außerdem ist es möglich, zusätzlich noch die Veranstaltung Mathe 2 zu hören. (Mathematikstudenten: siehe unten).
Beachte, dass man auch pädagogische Studien schon im ersten Semester besuchen kann. Falls sich Veranstaltungen aus dem zweiten
Fach mit Informatik überschneiden, kann man v.a. Informatik der Systeme aus dem vierten Semester tauschen.
Wenn du dir noch einen genaueren Überblick über den Studienverlauf verschaffen möchtest, dann schau doch mal ins Modulhandbuch.
%, zum Beispiel "`Einführung in die Schulpädagogik"' (Mi., 8-10, Kupferbau HS 25).

\noindent\makebox[\textwidth][c]{%
	\setlength{\fboxrule}{4pt}
	\fcolorbox{red}{white}{
		\begin{minipage}[t]{
				\textwidth}\textbf{Achtung!} Aufgrund der aktuellen Lage bezüglich COVID-19 können sich die Vorlesungstermine für dieses Semester noch ändern. Termine für Vorlesungen, die mit einem $^*$ markiert sind, sind unter Vorbehalt. Schaut am besten auf Alma (\url{https://alma.uni-tuebingen.de/}), ob die Termine dort geupdatet wurden.
\end{minipage}}}

\begin{center}
	\begin{tabular}{|c|c|c|c|c|c|} \hline
		Zeit      & 	Montag 		& Dienstag			& Mittwoch 			& Donnerstag 		& Freitag	 \\\hline\hline
		08 -- 09  &					& 					&					& 					&			\\\hline
		09 -- 10  & 				& 					& 					& 					&			\\\hline
		10 -- 11  &					& 					&					& 					&			\\\hline
		11 -- 12  & 				&  					&					&			 		& 			\\\hline
		12 -- 13  & 				& 				 	& 				    & 					& 			 \\\hline
		13 -- 14  & 				& 					& 					& 					& 			 \\\hline
		14 -- 15  & 				& Informatik II 	& 					& Informatik II 	& 				\\\cline{1-2}\cline{4-4}\cline{6-6}
		15 -- 16  &					& (\Infoprof) 		& 					& (\Infoprof) 		& 				\\\hline
		16 -- 17  & 				& 					& 					& 					&\\\hline
		17 -- 18  & 				& 					& 					& 					& \\
		\hline
	\end{tabular}

%	\scriptsize TI = Technische Informatik

\end{center}

Dieser Plan gilt für das erste Semester Informatik. Es kommen noch jeweils zwei Übungsstunden zu den Vorlesungen dazu.
Die Zeiten für die Übungsgruppen werden innerhalb der ersten Woche in den Vorlesungen bekannt gegeben.\\
%Falls du \textbf{im zweiten Fach Mathematik oder Physik} studierst: Die Inhalte der Mathematik 1 für Informatiker werden bereits durch andere Fächer aus der Mathematik oder der Physik abgedeckt, deshalb muss Mathematik 1 nicht belegt werden.
%Anstatt dieser Veranstaltung gibt es ein sog. Ausgleichsmodul, welches für eine andere Prüfungsleistung mit gleichem Aufwand steht. Als Ausgleichsmodul kannst du jedes Wahlpflichtmodul mit gleicher Anzahl an LP wählen, welches in Campus angeboten wird. Dieses ist für das 5. Semester vorgesehen.
%Für die Kombination mit Mathematik wird Lineare Algebra 1, Analysis 1 und Informatik 1 im ersten Semester empfohlen.
