Wie du dem folgenden Stundenplan entnehmen kannst, enthält das Lehramtsstudium der Informatik im ersten Semester zunächst nur die Informatik II-Vorlesung. Hinzu kommt noch das Seminar Fachdidaktik I, welches zwischen dem 01.09 und 05.09 als Blockveranstaltung stattfindet.
Beachte, dass man auch pädagogische Studien schon im ersten Semester besuchen kann. Falls sich Veranstaltungen aus dem zweiten
Fach mit Informatik überschneiden, kann man v.a. Informatik der Systeme aus dem vierten Semester tauschen.
Wenn du dir noch einen genaueren Überblick über den Studienverlauf verschaffen möchtest, dann schau doch mal ins Modulhandbuch.

\begin{center}
	\begin{tabular}{|c|c|c|c|}
		\hline
		Zeit     & Dienstag                   & Mittwoch & Donnerstag               \\ \hline
		08 -- 09 &                            &          &                          \\ \hline
		09 -- 10 &                            &          &                          \\ \hline
		10 -- 11 &                            &          &                          \\ \hline
		11 -- 12 &                            &          &                          \\ \hline
		12 -- 13 &                            &          &                          \\ \hline
		13 -- 14 &                            &          &                          \\ \hline
		14 -- 15 & Praktische Informatik II   &          & Praktische Informatik II \\ \cline{1-1} \cline{3-3}
		15 -- 16 & (\Infoprof)                &          & (\Infoprof)              \\ \hline
		16 -- 17 &                            &          &           \\ \hline
		17 -- 18 &                            &          &     \\ \hline
		\end{tabular}

\end{center}

\textbf{Info 2}:\\
Vollständiger Name: Praktische Informatik 2: Imperative und objektorientierte Programmierung\\
Dieses Semester wird die Vorlesung im Hörsaal N7 auf der Morgenstelle stattfinden.
Zusätzlich werden Tutorien stattfinden.
Die Zeiten für die Tutorien werden innerhalb der ersten Woche in den Vorlesungen bekannt gegeben.\\
\\