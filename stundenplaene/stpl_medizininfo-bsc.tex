Wie Ihr dem folgenden Stundenplan entnehmen könnt, enthält das Medizininformatik-Studium im ersten Semester neben der Informatik- und Mathematikvorlesung auch eine gute Portion Humanbiologie, außerdem dürft ihr Euch mit grundlegenden Problemen der IT im Gesundheitswesen beschäftigen sowie Euch grundlegende medizinische Fachbegriffe aneignen.\\
\fcolorbox{red}{white}{
	\begin{minipage}[t]{
			\textwidth}\textbf{Achtung!} Aufgrund der aktuellen Lage bezüglich COVID-19 können sich die Vorlesungstermine und das Format (online/präsenz) für dieses Semester noch ändern. Schaut am besten auf Alma (\url{https://alma.uni-tuebingen.de/alma/pages/cm/exa/coursecatalog/showCourseCatalog.xhtml?_flowId=showCourseCatalog-flow&_flowExecutionKey=e1s1}), ob die Termine dort geupdatet wurden.
\end{minipage}}

\begin{minipage}{\textwidth}
    \footnotesize
\begin{center}
\begin{tabular}{|c|c|c|c|c|} 
	\hline
	Zeit    & Montag       & Dienstag             & Mittwoch                          & Donnerstag                          \\ 
	\hline\hline
	08 – 09 & Mathematik I &                      & Mathematik I                      &                                   \\ 
	\cline{1-1}\cline{3-3}\cline{5-5}
	09 – 10 & (\Matheprof)      &                      & (\Matheprof), TBA                           &   \\ 
	\hline
	10 – 11 &              &                      &     Grundlagen der Medizininf.                              &                                   \\ 
	\cline{1-1}\cline{3-5}
	11 – 12 &              &                      &    (Dr. Lautenbacher)             &                                   \\ 
	\hline
	12 – 13 &              &                      &                                   &                                   \\ 
	\hline
	13 – 14 &              &                      &                                   &                                   \\ 
	\hline
	14 – 15 &              & Informatik I  &  &      Informatik I                             \\ 
	\cline{1-2}\cline{4-5}
	15 – 16 &              & (\Infoprof)              &          &  (\Infoprof)                \\ 
	\cline{1-3}\cline{5-5}
	16 – 17 &              &                      &                                   &                                   \\ 
	\hline
	17 – 18 &              &   Medizinische Terminologie    &  &                                   \\
	\hline
\end{tabular}
    ~\\
\scriptsize %\\
%HSM = gr. Hörsaal Medizinische Klinik (Uni-Kliniken Berg); H2C14 = Morgenstelle, H-Bau Erdgeschoss\\
%1: beginnt immer s.t., findet erst ab der zweiten Vorlesungswoche statt\\
\end{center}
\end{minipage}
Dieser Plan gilt für das erste Semester Medizininformatik. Es kommen noch jeweils zwei Übungsstunden für die Vorlesungen 
Informatik I und Mathematik I dazu. Die Zeiten für die Übungsgruppen werden innerhalb der ersten Woche in den Vorlesungen bekannt gegeben.
