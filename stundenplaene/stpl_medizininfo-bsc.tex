Wie Ihr dem folgenden Stundenplan entnehmen könnt, enthält das Medizininformatik-Studium im ersten Semester neben der Informatik- und Mathematikvorlesung auch eine gute Portion Biologie, außerdem dürft ihr Euch mit grundlegenden Problemen der IT im Gesundheitswesen beschäftigen sowie Euch grundlegende medizinische Fachbegriffe aneignen.

\begin{minipage}{\textwidth}
    \footnotesize
\begin{center}
	\begin{tabular}{|c|c|c|c|c|} \hline
		Zeit      & 			Montag 		& Dienstag			& Mittwoch 						& Donnerstag 			 \\
		\hline\hline
		08 -- 09  & 		Mathematik I 	&  					& Mathematik I 					&  						\\
		\cline{1-1}\cline{3-3}\cline{5-5}			
		09 -- 10  & 	(\Matheprof), N6    & 					& (\Matheprof), N6 				 &  						\\
		\hline
		10 -- 11  &							&					& Grundlagen der Medizininf.	& Med. Terminologie		\\
		\cline{1-3}
		11 -- 12  & 						&  					& (Lautenbacher), C9A03			&(Schneider), H2C14		\\
		\hline
		12 -- 13 & 							& 				 	& 				    			& 						 \\
		\hline
		13 -- 14 & 							& 					& 								& 						 \\
		\hline
		14 -- 15 & 							& Informatik I 		& 								& Informatik I 			\\
		\cline{1-2}\cline{4-4}
		15 -- 16 &							& (\Infoprof), N6 	& 								& (\Infoprof), N6 		\\
		\hline
		16 -- 17 & Humanbiologie I$^{1)}$	& Humanbiologie I   & & \\
		\cline{1-1}\cline{4-5}
		17 -- 18 & (Huber), ?				& (Huber), ?		& &  \\
		\hline
	\end{tabular}
    ~\\
\scriptsize	
%HSM = gr. Hörsaal Medizinische Klinik (Uni-Kliniken Berg); H2C14 = Morgenstelle, H-Bau Erdgeschoss\\
1: beginnt immer s.t., findet erst ab der zweiten Vorlesungswoche statt\\
\end{center}
\end{minipage}
Dieser Plan gilt für das erste Semester Medizininformatik. Es kommen noch jeweils zwei Übungsstunden für die Vorlesungen 
Informatik I und Mathematik I dazu. Die Zeiten für die Übungsgruppen werden innerhalb der ersten Woche in den Vorlesungen bekannt gegeben.
