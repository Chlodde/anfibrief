Wie Ihr dem folgenden Stundenplan entnehmen könnt, enthält das Medizininformatik-Studium im ersten Semester neben der Informatik- und Mathematikvorlesung auch eine gute Portion Biologie, außerdem dürft ihr Euch mit grundlegenden Problemen der IT im Gesundheitswesen beschäftigen sowie Euch grundlegende medizinische Fachbegriffe aneignen.\\
\fcolorbox{red}{white}{
		\begin{minipage}[t]{
			\textwidth}\textbf{Achtung!} Aufgrund der aktuellen Lage bezüglich COVID-19 können sich die Vorlesungstermine für dieses Semester noch ändern. Termine für Vorlesungen, die mit einem $^*$ markiert sind, sind unter Vorbehalt. Schaut am besten auf Alma (\url{https://alma.uni-tuebingen.de/alma/pages/cm/exa/coursecatalog/showCourseCatalog.xhtml?_flowId=showCourseCatalog-flow&_flowExecutionKey=e1s1}), ob die Termine dort geupdatet wurden.
		\end{minipage}}

\begin{minipage}{\textwidth}
    \footnotesize
\begin{center}
	\begin{tabular}{|c|c|c|c|c|c|c|}
	\hline
	 Zeit     &    Montag                    & Dienstag          & Mittwoch               	  	& Donnerstag & Freitag 	& Samstag\\ \hline\hline
	 08 -- 09 &    Mathematik I              &                   & Mathematik I      	  	&  &  			&\\ \cline{1-1} \cline{3-3} \cline{5-7} 
	 09 -- 10 &    (\Matheprof), TBA         &                   & (\Matheprof), TBA 	  	&  &  		       	& Medizinische Terminologie$^{*2}$\\ \hline
	 10 -- 11 &                              &                   &                   	  	&  &  			&\\ \cline{1-1} \cline{3-7} 
	 11 -- 12 &                              &                   &                   	  	&  &  			&\\ \hline
	 12 -- 13 &                              &                   &                   	  	&  &  			&\\ \hline
	 13 -- 14 &                              &                   &                   	  	&  &  			&\\ \hline
	 14 -- 15 &                              & Informatik I$^{*3}$  & Grundlagen der Medizininf.$^{*1}$&  &  			&\\ \cline{1-2} \cline{4-7} 
	 15 -- 16 &                              & (\Infoprof), TBA  & (Dr. Lautenbacher), TBA    	&  &  			&\\ \cline{1-3} \cline{5-7}
	 16 -- 17 &                              &                   &                   	  	&  &  			&\\ \hline
	 17 -- 18 &                              &                   & Grundlagen der Medizininf.$^{*1}$&  &  			&\\ \hline
	\end{tabular}
    ~\\
\scriptsize %\\
1: Vorraussichtlicher Termin: 15.30--16.30 Uhr und 17-18 Uhr s.t.\\
2: Vorlesung geht von 9.30 -- 10.30 Uhr\\
3: Vorlesung findet dieses Semester asynchron, also online in Videoformat statt. (Link folgt) Übungen werden sowohl online als auch präsent angeboten.
%HSM = gr. Hörsaal Medizinische Klinik (Uni-Kliniken Berg); H2C14 = Morgenstelle, H-Bau Erdgeschoss\\
%1: beginnt immer s.t., findet erst ab der zweiten Vorlesungswoche statt\\
\end{center}
\end{minipage}
Dieser Plan gilt für das erste Semester Medizininformatik. Es kommen noch jeweils zwei Übungsstunden für die Vorlesungen 
Informatik I und Mathematik I dazu. Die Zeiten für die Übungsgruppen werden innerhalb der ersten Woche in den Vorlesungen bekannt gegeben.
