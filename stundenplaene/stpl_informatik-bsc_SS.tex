Wie Ihr dem folgenden Stundenplan entnehmen könnt, enthält das Informatik-Studium im ersten
Semester neben der Informatikvorlesung auch eine gute Portion Mathe.

\begin{center}
    % Reguläres Semester

    %\begin{tabular}{|c|c|c|c|c|c|}
    %    \hline
    %    Zeit     & Montag                   & Dienstag                   & Mittwoch      & Donnerstag               & Freitag \\ \hline
    %    08 -- 09 &                          & Technische                 &               &                          &         \\ \cline{1-2}
    %    09 -- 10 &                          & Informatik II              &               &                          &         \\ \cline{1-2}
    %    10 -- 11 & Mathematik II            & (Prof. Menth)              & Mathematik II &                          &         \\ \cline{1-1}
    %    11 -- 12 & (\Matheprof)             &                            & (\Matheprof)  &                          &         \\ \hline
    %    12 -- 13 &                          &                            &               &                          &         \\ \hline
    %    13 -- 14 &                          &                            &               &                          &         \\ \hline
    %    14 -- 15 &                          & Praktische Informatik II   &               & Praktische Informatik II &         \\ \cline{1-2}
    %    15 -- 16 &                          & (\Infoprof)                &               & (\Infoprof)              &         \\ \hline
    %    16 -- 17 & Technische Informatik II &                            &               &                          &         \\ \hline
    %    17 -- 18 & (Prof. Menth)            &                            &               &                          &         \\ \hline
    %    \end{tabular}

    % Corona Semester
    \begin{tabular}{|c|c|c|c|c|}
        \hline
        Zeit     & Montag                   & Dienstag                   & Mittwoch & Donnerstag                 \\ \hline
        08 -- 09 &  & Technische & \begin{tabular}[c]{@{}c@{}}Mathematik II\\ (\Matheprof)\end{tabular} &  \\ \cline{1-2} \cline{4-5} 
        09 -- 10 &                          & Informatik II              &          &                            \\ \cline{1-2} \cline{4-5} 
        10 -- 11 &                          & (Prof. Menth)              &          &                            \\ \cline{1-2} \cline{4-5} 
        11 -- 12 &                          &                            &          &                            \\ \hline
        12 -- 13 &                          &                            &          &                            \\ \hline
        13 -- 14 &                          &                            &          &                            \\ \hline
        14 -- 15 &                          & Praktische Informatik II   &          & Praktische Informatik II   \\ \cline{1-2} \cline{4-4}
        15 -- 16 &                          & (\Infoprof)                &          & (\Infoprof)                \\ \hline
        16 -- 17 & Technische Informatik II &                            &          &                            \\ \hline
        17 -- 18 & (Prof. Menth)            &                            &          &                            \\ \hline
        \end{tabular}

\end{center}

Dieser Plan gilt für das zweite Semester Informatik, für euch ist es jedoch das erste.\\
\textbf{Mathe II}:\\
Vollständiger Name: Mathematik für Informatik 2: Lineare Algebra\\
Dieses Semester wird die Vorlesung und die Tutorien voraussichtlich in Präsenz angeboten. Zusätzlich findet ab dem 25.04. montags das Rechenzentrum statt, wo Studierende gemeinsam die Aufgaben und Inhalte der Vorlesung diskutieren können, wobei auch Tutorinnen und Tutoren anwesend sein werden.Die Teilnahme am Rechenzentrum ist freiwillig.\\

\textbf{Info II}:\\
Vollständiger Name: Praktische Informatik 2: Imperative und objekt-orientierte Programmierung\\
Dieses Semester wird die Vorlesung im Hybrid-Modell stattfinden. Zusätzlich werden Tutorien in Präsenz stattfinden.\\

Die Zeiten für die Tutorien werden innerhalb der ersten Woche in den Vorlesungen bekannt gegeben.
