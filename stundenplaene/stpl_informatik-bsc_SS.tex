Wie du dem folgenden Stundenplan entnehmen kannst, enthält das Informatik-Studium im ersten
Semester neben der Informatikvorlesung auch eine gute Portion Mathe.
\vspace{-0.5cm}
\begin{center}
    % Reguläres Semester
    \resizebox{\textwidth}{!}{
    \begin{tabular}{|c|c|c|c|c|c|}
       \hline
       Zeit     & Montag                   & Dienstag                   & Mittwoch      & Donnerstag               & Freitag \\ \hline
       08 -- 09 &                          & Technische Informatik II   &               &                          &         \\ \cline{1-2} \cline{4-6}
       09 -- 10 &                          & (Prof. Menth)              &               &                          &         \\ \hline
       10 -- 11 & Mathematik II            &                            & Mathematik II &                          &         \\ \cline{1-1} \cline{3-3} \cline{5-6}
       11 -- 12 & (\Matheprof)             &                            & (\Matheprof)  &                          &         \\ \hline
       12 -- 13 &                          &                            &               &                          &         \\ \cline{1-3} \cline{5-6}
       13 -- 14 &                          &                            & Mathe         &                          &         \\ \cline{1-3} \cline{5-6}
       14 -- 15 &                          & Praktische Informatik II   & Rechenzentrum & Praktische Informatik II &         \\ \cline{1-2} \cline{6-6}
       15 -- 16 &                          & (\Infoprof)                & (ab 26.04.)   & (\Infoprof)              &         \\ \cline{1-3} \cline{5-6}
       16 -- 17 & Technische               &                            &               &                          &         \\ \cline{1-1} \cline{3-3} \cline{5-6}
       17 -- 18 & Informatik II            &                            &               &                          &         \\ \cline{1-1} \cline{3-6}
       18 -- 19 & (Prof. Menth)            &                            &               &                          &         \\ \hline
    \end{tabular}
    }

    % Corona Semester
    % \resizebox{\textwidth}{!}{% <- % is important
    % \begin{tabular}{|c|cc|c|c|c|}
    %     \hline
    %     Zeit     & \multicolumn{2}{c|}{Montag}                        & Dienstag                   & Mittwoch                    & Donnerstag                 \\ \hline
    %     08 -- 09 & \multicolumn{2}{c|}{}                              &                            &                             &                            \\ \hline
    %     09 -- 10 & \multicolumn{2}{c|}{}                              &                            &                             &                            \\ \hline
    %     10 -- 11 & \multicolumn{2}{c|}{Mathematik II}                 &                            & Mathematik II               &                            \\ \cline{1-1} \cline{4-4} \cline{6-6}
    %     11 -- 12 & \multicolumn{2}{c|}{(\Matheprof)}   &                            & (\Matheprof) &                            \\ \hline
    %     12 -- 13 & \multicolumn{2}{c|}{}                              &                            &                             &                            \\ \hline
    %     13 -- 14 & \multicolumn{2}{c|}{}                              &                            &                             &                            \\ \hline
    %     14 -- 15 & \multicolumn{2}{c|}{}                              & Praktische Informatik II   &                             & Praktische Informatik II   \\ \cline{1-3} \cline{5-5}
    %     15 -- 16 & \multicolumn{2}{c|}{}                              & (\Infoprof)                &                             & (\Infoprof)                \\ \hline
    %     \multicolumn{1}{|l|}{16 -- 17} & \multicolumn{1}{c|}{} & Technische Informatik II & \multicolumn{1}{l|}{} & \multicolumn{1}{l|}{} & \multicolumn{1}{l|}{} \\ \cline{1-2} \cline{4-6}
    %     17 -- 18 & \multicolumn{1}{c|}{Mathe}         & (Prof. Menth) &                            &                             &                            \\ \cline{1-1} \cline{3-6}
    %     18 -- 19 & \multicolumn{1}{c|}{Rechenzentrum} &               &                            &                             &                            \\ \cline{1-1} \cline{3-6}
    %     19 -- 20 & \multicolumn{1}{c|}{(ab 25.04)}    &               &                            &                             &                            \\ \hline
    %     \end{tabular}% <- % is important
    % }
\end{center}
\noindent Dieser Plan gilt für das zweite Semester Informatik, für euch ist es jedoch das erste.\\

\textbf{Mathe 2}:\\
Vollständiger Name: Mathematik für Informatik 2: Lineare Algebra\\
Dieses Semester wird die Vorlesung im Hörsaal N7 auf der Morgenstelle stattfinden.
Zusätzlich zur Vorlesung findet ein Tutorium statt. \\
Ab dem 26.04. findet mittwochs das Rechenzentrum statt. Hier können Studierende gemeinsam die Aufgaben und Inhalte der Vorlesung diskutieren,
wobei auch Tutorinnen und Tutoren anwesend sein werden. Die Teilnahme am Rechenzentrum ist freiwillig.\\

\textbf{Info 2}:\\
Vollständiger Name: Praktische Informatik 2: Imperative und objektorientierte Programmierung\\
Dieses Semester wird die Vorlesung wieder in Präsenz im Hörsaal N7 auf der Morgenstelle stattfinden.
Zusätzlich werden Tutorien in Präsenz stattfinden.
Die Zeiten für die Tutorien werden innerhalb der ersten Woche in den Vorlesungen bekannt gegeben.\\
% Es wird für die Mathe II und die Info II auch noch freiwillige Zusatz-Tutorium und Einführungskurse geben.\\

\textbf{üBK}:\\
Im Studienplan wird empfohlen eine übK\footnote{überfachlich berufsfeldorientierte Kompetenze, in Zukunft auch Liberal Education genannt}
mit 3 LP zu belegen. Hier ist dir völlig freigestellt was du belegst, solange es mit ECTS Punkten versehen, benotet und kein Sport ist.
Veranstaltung von gewissen Fächern (u.a. Medizin, Phsychologie) lassen sich jedoch nur hören, wenn du dieses als Schwerpunktfach gewählt hast.
Hierdurch verlierst du jedoch die Freiheit der übK und darfst nur Vorlesungen aus diesen Fächern hören. Näheres erfährst du bei der fachspezifischen Begrüßung.

