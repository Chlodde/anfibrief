Wie du dem folgenden Stundenplan entnehmen kannst, enthält das Bioinformatik-Studium im ersten
Semester neben den Informatik- und Biochemie-Vorlesungen auch eine gute Portion Mathe.
%\vspace{-1cm}
\begin{center}
    \resizebox{\textwidth}{!}{% <- % is important
        \begin{tabular}{|c|c|c|c|c|c|}
        \hline
        Zeit     & Montag        & Dienstag                 & Mittwoch      & Donnerstag               & Freitag            \\ \hline
        08 -- 09 &               	&                          &               &                          &                    \\ \hline
        09 -- 10 &               	&                          &               &                          &                    \\ \hline
        10 -- 11 & Mathematik II &                          & Mathematik II &                          & Biochemie          \\ \cline{1-1} \cline{3-3} \cline{5-5}
        11 -- 12 & (\Matheprof)  &                          & (\Matheprof)  & Biochemie                & (Prof. Nürnberger) \\ \hline
        12 -- 13 &              	 	&                          &               &                          &                    \\ \hline
        13 -- 14 &               	&                          &          &Einf. in die Bioinformatik\footnotemark &                    \\ \cline{1-4} \cline{5-6}
        14 -- 15 &              		 & Praktische Informatik II & Mathe  & Praktische Informatik II &                    \\ \cline{1-2} \cline{6-6}
        15 -- 16 &               	& (\Infoprof)              &  Rechenzentrum  & (\Infoprof)              &                    \\ \cline{1-3} \cline{5-6}
        16 -- 17 &               	&                          &  (ab 23.04.) &                          &                    \\ \cline{1-3} \cline{5-6}
        17 -- 18 &               	&                          &               &                          &                    \\ \hline
        %18 -- 19 &               &                          &               &                          &                    \\ \cline{1-1} \cline{3-6}
        %19 -- 20 &               &                          &               &                          &                    \\ \hline
        \end{tabular}% <- this one aswell
    }
\end{center}
\footnotetext{Einführung in die Bioinformatik (Prof. Thiel)}
\noindent Dieser Plan gilt für das zweite Semester Bioinformatik, für euch ist es jedoch das Erste.\\

\textbf{Mathe 2}:\\
Vollständiger Name: Mathematik für Informatik 2: Lineare Algebra\\
Dieses Semester wird die Vorlesung im Hörsaal N7 auf der Morgenstelle stattfinden.
Zusätzlich zur Vorlesung findet ein Tutorium statt. \\
Ab dem 24.04. findet mittwochs das Rechenzentrum statt. Hier können Studierende gemeinsam die Aufgaben und Inhalte der Vorlesung diskutieren,
wobei auch Tutorinnen und Tutoren anwesend sein werden. Die Teilnahme am Rechenzentrum ist freiwillig.\\

\textbf{Info 2}:\\
Vollständiger Name: Praktische Informatik 2: Imperative und objektorientierte Programmierung\\
Dieses Semester wird die Vorlesung im Hörsaal N7 auf der Morgenstelle stattfinden.
Zusätzlich werden Tutorien stattfinden.
Die Zeiten für die Tutorien werden innerhalb der ersten Woche in den Vorlesungen bekannt gegeben.\\
% Es wird für die Mathe II und die Info II auch noch freiwillige Zusatz-Tutorium und Einführungskurse geben.\\

\textbf{üBK}:\\
Im Studienplan wird empfohlen übK\footnote{überfachlich berufsfeldorientierte Kompetenzen, in Zukunft auch Liberal Education genannt}
mit (insgesamt) 3-6 LP zu belegen. Hier ist dir völlig freigestellt was du belegst, solange es mit ECTS Punkten versehen, benotet und kein Sport ist.\\

\pagebreak
