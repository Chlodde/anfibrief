Wie Ihr dem folgenden Stundenplan entnehmen könnt, enthält das Bioinformatik-Studium im ersten
Semester neben den Informatik- und Biologie-Vorlesungen auch eine gute Portion Mathe und Chemie.
Das ist zwar etwas anstrengend, aber mit der richtigen Planung und unseren Tipps ist es gut
machbar.


\begin{center}
    % Normales SoSe

    % TODO Biochemie hinzufügen (siehe corona)
    %\begin{tabular}{|c|c|c|c|c|c|}
    %    \hline
    %    Zeit     & Montag         & Dienstag                   & Mittwoch       & Donnerstag                 & Freitag \\ \hline
    %    08 -- 09 &                &                            &                &                            &         \\ \hline
    %    09 -- 10 &                &                            &                &                            &         \\ \hline
    %    10 -- 11 & Mathematik II  &                            & Mathematik II  &                            &         \\ \cline{1-1}
    %    11 -- 12 & (\Matheprof)   &                            & (\Matheprof)   &                            &         \\ \hline
    %    12 -- 13 &                &                            &                &                            &         \\ \hline
    %    13 -- 14 &                &                            &                &                            &         \\ \hline
    %    14 -- 15 &                & Praktische Informatik II   &                & Praktische Informatik II   &         \\ \cline{1-2}
    %    15 -- 16 &                & (\Infoprof)                &                & (\Infoprof)                &         \\ \hline
    %    16 -- 17 &                &                            &                &                            &         \\ \hline
    %    17 -- 18 &                &                            &                &                            &         \\ \hline
    %    \end{tabular}

    % Corona  Plan
    \begin{tabular}{|c|c|c|c|c|}
        \hline
        Zeit     & Montag                   & Dienstag                   & Mittwoch & Donnerstag                 \\ \hline
        08 -- 09 &  & Technische & \begin{tabular}[c]{@{}c@{}}Mathematik II\\ (\Matheprof)\end{tabular} &  \\ \cline{1-2} \cline{4-5} 
        09 -- 10 &                          & Informatik II              &          &                            \\ \cline{1-2} \cline{4-5} 
        10 -- 11 &                          & (Prof. Menth)              &          &                            \\ \cline{1-2} \cline{4-5} 
        11 -- 12 &                          &                            &          &                            \\ \hline
        12 -- 13 &                          &                            &          &                            \\ \hline
        13 -- 14 &                          &                            &          &                            \\ \hline
        14 -- 15 &                          & Praktische Informatik II   &          & Praktische Informatik II   \\ \cline{1-2} \cline{4-4}
        15 -- 16 &                          & (\Infoprof)                &          & (\Infoprof)                \\ \hline
        16 -- 17 & Technische Informatik II &                            &          &                            \\ \hline
        17 -- 18 & (Prof. Menth)            &                            &          &                            \\ \hline
        \end{tabular}
\end{center}


Dieser Plan gilt für das zweite Semester Bioinformatik, für euch ist es jedoch das Erste.\\

\textbf{Mathe II}:\\
Vollständiger Name: Mathematik für Informatik 2: Lineare Algebra\\
Dieses Semester wird die Vorlesung als Video hochgeladen. Am Mittwoch gibt es eine Live-Fragestunde. Zusätzlich werden Tutorien in Präsenz stattfinden.

\textbf{Info II}:\\
Vollständiger Name: Praktische Informatik 2: Imperative und objekt-orientierte Programmierung\\
Dieses Semester wird die Vorlesung im Hybrid-Modell stattfinden. Zusätzlich werden Tutorien in Präsenz stattfinden.\\

Die Zeiten für die Übungsgruppen werden innerhalb der ersten Woche in den Vorlesungen bekannt gegeben.\\
Es wird für die Mathe II und die Info II auch noch freiwillige Zusatz-Tutorium und Einführungskurse geben.
Im Studienplan wird empfohlen in eurem ersten Semester eine übK\footnote{überfachlichen Kompetenzen} (6LP) zu belegen. Hier ist euch völlig freigestellt was ihr belegt, solange es mit ECTS Punkten versehen und mit einer Note bewertet wird (kein Sport!).
