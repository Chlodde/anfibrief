Wie Ihr dem folgenden Stundenplan entnehmen könnt, enthält das Bioinformatik-Studium im ersten
Semester neben den Informatik- und Biologie-Vorlesungen auch eine gute Portion Mathe und Chemie.
\vspace{-1cm}
\begin{center}
    % Normales SoSe
    % TODO Biochemie hinzufügen (siehe corona)
    %\resizebox{\textwidth}{!}{% <- % is important
    %\begin{tabular}{|c|c|c|c|c|c|}
    %    \hline
    %    Zeit     & Montag         & Dienstag                   & Mittwoch       & Donnerstag                 & Freitag \\ \hline
    %    08 -- 09 &                &                            &                &                            &         \\ \hline
    %    09 -- 10 &                &                            &                &                            &         \\ \hline
    %    10 -- 11 & Mathematik II  &                            & Mathematik II  &                            &         \\ \cline{1-1}
    %    11 -- 12 & (\Matheprof)   &                            & (\Matheprof)   &                            &         \\ \hline
    %    12 -- 13 &                &                            &                &                            &         \\ \hline
    %    13 -- 14 &                &                            &                &                            &         \\ \hline
    %    14 -- 15 &                & Praktische Informatik II   &                & Praktische Informatik II   &         \\ \cline{1-2}
    %    15 -- 16 &                & (\Infoprof)                &                & (\Infoprof)                &         \\ \hline
    %    16 -- 17 &                &                            &                &                            &         \\ \hline
    %    17 -- 18 &                &                            &                &                            &         \\ \hline
    %    \end{tabular}% <- this one aswell
%}

    % Corona  Plan
    \resizebox{\textwidth}{!}{% <- % is important
        \begin{tabular}{|c|c|c|c|c|c|}
        \hline
        Zeit     & Montag        & Dienstag                 & Mittwoch      & Donnerstag               & Freitag            \\ \hline
        08 -- 09 &               &                          &               &                          &                    \\ \hline
        09 -- 10 &               &                          &               &                          & Biochemie          \\ \cline{1-5}
        10 -- 11 & Mathematik II &                          & Mathematik II &                          & (Prof. Nürnberger) \\ \cline{1-1} \cline{3-3} \cline{5-6} 
        11 -- 12 & (\Matheprof)  &                          & (\Matheprof)  & Biochemie                &                    \\ \hline
        12 -- 13 &               &                          &               &                          &                    \\ \hline
        13 -- 14 &               &                          &               &                          &                    \\ \hline
        14 -- 15 &               & Praktische Informatik II &               & Praktische Informatik II &                    \\ \cline{1-2} \cline{4-4} \cline{6-6}
        15 -- 16 &               & (\Infoprof)              &               & (\Infoprof)              &                    \\ \hline
        16 -- 17 &               &                          &               &                          &                    \\ \hline
        17 -- 18 & Mathe         &                          &               &                          &                    \\ \cline{1-1} \cline{3-6}
        18 -- 19 & Rechenzentrum &                          &               &                          &                    \\ \cline{1-1} \cline{3-6}
        19 -- 20 & (ab 25.04)    &                          &               &                          &                    \\ \hline
        \end{tabular}% <- this one aswell
}
\end{center}
\noindent Dieser Plan gilt für das zweite Semester Bioinformatik, für euch ist es jedoch das Erste.\\

\textbf{Mathe II}:\\
Vollständiger Name: Mathematik für Informatik 2: Lineare Algebra\\
Dieses Semester wird die Vorlesung voraussichtlich in Präsenz im Hörsaal N7 auf der Morgenstelle stattfinden.
Zusätzlich zur Vorlesung findet ein Tutorium statt. \\
Ab dem 25.04. findet montags das Rechenzentrum statt. Hier können Studierende gemeinsam die Aufgaben und Inhalte der Vorlesung diskutieren,
wobei auch Tutorinnen und Tutoren anwesend sein werden. Die Teilnahme am Rechenzentrum ist freiwillig.\\

\textbf{Info II}:\\
Vollständiger Name: Praktische Informatik 2: Imperative und objekt-orientierte Programmierung\\
Dieses Semester wird die Vorlesung im Hybrid-Modell stattfinden. Zusätzlich werden Tutorien in Präsenz stattfinden.
Die Zeiten für die Tutorien werden innerhalb der ersten Woche in den Vorlesungen bekannt gegeben.
Es wird für die Mathe II und die Info II auch noch freiwillige Zusatz-Tutorium und Einführungskurse geben.\\

\textbf{üBK}:\\
Im Studienplan wird empfohlen eine übK\footnote{überfachlich berufsfeldorientierte Kompetenze, in Zukunft auch Liberal Education genannt}
mit (insgesamt) 6LP zu belegen. Hier ist dir völlig freigestellt was du belegst, solange es mit ECTS Punkten versehen, benotet und kein Sport ist.\\
