Wie Ihr dem folgenden Stundenplan entnehmen könnt, enthält das Bioinformatik-Studium im ersten
Semester neben den Informatik- und Biologie-Vorlesungen auch eine gute Portion Mathe und Chemie.
Das ist zwar etwas anstrengend, aber mit der richtigen Planung und unseren Tipps ist es gut
machbar.


\begin{center}
\begin{tabular}{|c|c|c|c|c|c|} \hline
Zeit & Montag & Dienstag & Mittwoch & Donnerstag & Freitag \\
\hline\hline
08 -- 09  & & & & &\\
\hline
09 -- 10  & &  & & & \\
\hline
10 -- 11  & Mathematik II & & Mathematik II & &\\
\cline{1-1}\cline{3-3}\cline{5-6}
11 -- 12  & (\Matheprof) &  & (\Matheprof) & & \\
\hline
12 -- 13 & & & & &  \\
\hline
13 -- 14 & & & & & \\
\hline
14 -- 15 & & Praktische Informatik II & & Praktische Informatik II & \\
\cline{1-2}\cline{4-4}\cline{6-6}
15 -- 16 & & (\Infoprof) & & (\Infoprof) & \\
\hline
16 -- 17 & & & & &\\
\cline{1-6}
17 -- 18 & & & & & \\
\hline
\end{tabular}
\end{center}


Dieser Plan gilt für das zweite Semester Bioinformatik, für euch ist es jedoch das Erste. Es kommen noch einige Übungsstunden zu den einzelnen Vorlesungen dazu. Die Zeiten für die Übungsgruppen werden innerhalb der ersten Woche in den Vorlesungen bekannt gegeben.\\
Es wird für die Mathe II und die Info II auch noch freiwillige Zusatz-Tutorien für euch geben, damit euch der Einstieg etwas leichter fällt.\\
%Im Studienplan wird empfohlen in eurem ersten Semester eine Schlüsselqualifikation (3LP) zu belegen. Hier ist euch völlig freigestellt was ihr belegt, solange es mit ECTS Punkten versehen und mit einer Note bewertet wird (nur Sportkurse sind nicht erlaubt!).
%Chemie besteht aus den Teilen "`Anorganische Chemie"' bei Professor Schweda in der ersten Hälfte des Semesters und "`Organische Chemie"' bei Professor Speiser im zweiten Teil des Semesters.
%BMZ (Biomoleküle und Zelle) wird bis Ende November im Block unterrichtet, ab der zweiten Semesterwoche kommt noch ein Praktikum hinzu.
%, ab Mitte Januar dann Tierphysiologie. In den Semesterferien wird noch ein Tierphysiologisches Praktikum hinzu kommen.
