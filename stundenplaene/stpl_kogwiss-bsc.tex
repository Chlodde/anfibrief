Wie Ihr dem folgenden Stundenplan entnehmen könnt, enthält das Studium der Kognitionswissenschaft
im ersten Semester neben Veranstaltungen der Psychologie und Neurobiologie eine gute Portion Informatik und Mathe.\\


\begin{center}
\footnotesize
\begin{tabular}{|c|c|c|c|c|} \hline
Zeit      & 			Montag 		& Dienstag							& Mittwoch 			& Donnerstag  \\
\hline\hline
08:00 -- 08:30  & 		Mathematik I, 	&									& Mathematik I, 	& \\
\cline{1-1}\cline{3-3}\cline{5-5}
08:30 -- 09:00  & 	(\Matheprof)	 & 									& (\Matheprof) 	&  \\
\cline{1-1}\cline{3-3}\cline{5-5}
09:00 -- 09:30  & 		 & 									& 	&  \\
\cline{1-1}\cline{3-3}\cline{5-5}
09:30 -- 10:00 & & Neurobio\,u.\,Sinnesphy- & &\\
\cline{1-1}\cline{2-2}\cline{4-5}
10:00 -- 10:30 & &siologie (Mallot), N6 & & \\
\hline
10:30 -- 11:00 & & & & \\
\hline
11:00 -- 11:30 &  & Kognitionswissenschaft 1 & 					& Statistik I \\
\cline{1-1}\cline{2-2}\cline{4-4}
11:30 -- 12:00 & 	& \,(Rolke),\,online, s.t.	& 					& (Levina), online, s.t.\\
\hline
12:00 -- 12:30 & & & & \\
\hline
12:30 -- 13:00 & & & & \\
\hline
13:00 -- 13:30 & & & & \\
\hline
13:30 -- 14:00 & & & & \\
\hline
14:00 -- 14:30 &  						& Informatik I$^1$ 						&					 & Informatik I  \\
\cline{1-2}\cline{4-4}
14:30 -- 15:00 &  						&  (\Infoprof)  				&					& (\Infoprof)  \\
\cline{1-2}\cline{4-4}
15:00 -- 15:30 & & & &\\
\cline{1-2}\cline{4-4}
15:30 -- 16:00 & & & &\\
\hline
\end{tabular}\\
\scriptsize  N6 = Hörsaalzentrum Morgenstelle
%N1, N6, N7 = Hörsaalzentrum Morgenstelle; N12 = E-Bau Morgenstelle (E3A07); H4329 = Psychologisches Institut \\
%F119 = Sand 6, kleiner Hörsaal
\end{center}
1: Vorlesung findet dieses Semester nicht im Hörsaal, sondern über hochgeladene Videos auf Youtube statt. (Link folgt) Übungen werden sowohl online als auch präsent angeboten.\\
\\

Dieser Plan gilt vorläufig für das erste Semester Kognitionswissenschaft. Bitte informiert euch auf ALMA über Änderungen des Stundenplans aufgrund der derzeitigen Situation.
%(Die Veranstaltungen der Psychologie beginnen erst in der 2. Semesterwoche).
Es kommen noch jeweils Übungen zu den Vorlesungen Mathe\,I und Informatik\,I %und ein optionales Tutorium für Forschungsmethoden der Psychologie 
dazu.
%und optionale Tutorien für die Psychologievorlesungen dazu. 
Außerdem gibt es ein Seminar zur Vorlesung Neurobiologie und Sinnesphysiologie (Mo 14-16 Uhr oder Do 16-18 Uhr).

Die Zeiten für die Übungsgruppen werden innerhalb der ersten Woche in den Vorlesungen bekannt gegeben.
% Zusätzlich kann noch die Veranstaltung \emph{Introduction to Linguistics} aus dem 3. Semester vorgezogen werden (jeweils Di, Do und Fr von 8 bis 10 Uhr).
