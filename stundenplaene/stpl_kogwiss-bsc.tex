Wie Ihr dem folgenden Stundenplan entnehmen könnt, enthält das Studium der Kognitionswissenschaft
im ersten Semester neben Veranstaltungen der Psychologie und Neurobiologie eine gute Portion Informatik und Mathe.\\


\noindent\makebox[\textwidth][c]{%
	\setlength{\fboxrule}{4pt}
	\fcolorbox{red}{white}{
		\begin{minipage}[t]{
				\textwidth}\textbf{Achtung!} Aufgrund der aktuellen Lage bezüglich COVID-19 können sich die Vorlesungstermine für dieses Semester noch ändern. Termine für Vorlesungen, die mit einem $^*$ markiert sind, sind unter Vorbehalt. Schaut am besten auf Alma (\url{https://alma.uni-tuebingen.de/alma/pages/cm/exa/coursecatalog/showCourseCatalog.xhtml?_flowId=showCourseCatalog-flow&_flowExecutionKey=e1s1}), ob die Termine dort geupdatet wurden.
\end{minipage}}}

\begin{center}
\footnotesize
\begin{tabular}{|c|c|c|c|c|} \hline
Zeit      & 			Montag 		& Dienstag							& Mittwoch 			& Donnerstag  \\\hline\hline
08:00 -- 08:30  & 		Mathematik I, 	&									& Mathematik I, 	& \\
\cline{1-1}\cline{3-3}\cline{5-5}
08:30 -- 09:00  & 	(\Matheprof)	 & 									& (\Matheprof)	&  \\
\hline
09:00 -- 09:30  & 		 & 									& 	&  \\
\hline
09:30 -- 10:00 &  & & &\\
\hline
10:00 -- 10:30  &   Neuro-\,u.\,Sinnesphy- & Kognitionswissenschaft  1& & \\
\cline{1-1}\cline{4-5}
10:30 -- 11:00 &   siologie (Mallot)  &  \,(Rolke) & & \\
\hline
11:00 -- 11:30 &  &  & 					& Statistik I \\
\cline{1-4}
11:30 -- 12:00 & 	&	& 					& (Franz)\\
\hline
12:00 -- 12:30 & & & & \\
\hline
12:30 -- 13:00 & & & & \\
\hline
13:00 -- 13:30 & & & & \\
\hline
13:30 -- 14:00 & & & & \\
\hline
14:00 -- 14:30 &  						&	Informatik I					&					 &   Informatik I\\
\cline{1-2}\cline{4-4}
14:30 -- 15:00 &  						&   (\Infoprof)				&					&   (\Infoprof)	\\
\hline
15:00 -- 15:30 & & & &\\
\hline
15:30 -- 16:00 & & & &\\
\hline
\end{tabular}\\
\scriptsize  N6 = Hörsaalzentrum Morgenstelle
\end{center}


Dieser Plan gilt vorläufig für das erste Semester Kognitionswissenschaft. Bitte informiert euch auf ALMA über Änderungen des Stundenplans aufgrund der derzeitigen Situation.
%(Die Veranstaltungen der Psychologie beginnen erst in der 2. Semesterwoche).
Es kommen noch jeweils Übungen zu den Vorlesungen Mathe\,I und Informatik\,I %und ein optionales Tutorium für Forschungsmethoden der Psychologie 
dazu.
%und optionale Tutorien für die Psychologievorlesungen dazu. 
Außerdem gibt es ein Seminar zur Vorlesung Neurobiologie und Sinnesphysiologie (Mo 14-16 Uhr oder Do 14-16 Uhr).

Die Zeiten für die Übungsgruppen werden innerhalb der ersten Woche in den Vorlesungen bekannt gegeben.
% Zusätzlich kann noch die Veranstaltung \emph{Introduction to Linguistics} aus dem 3. Semester vorgezogen werden (jeweils Di, Do und Fr von 8 bis 10 Uhr).
