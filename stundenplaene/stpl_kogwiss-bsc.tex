Wie du dem folgenden Stundenplan entnehmen kannst, enthält das Studium der Kognitionswissenschaft
im ersten Semester neben Veranstaltungen der Psychologie und Neurobiologie eine gute Portion Informatik und Mathe.


\noindent\makebox[\textwidth][c]{%
	\setlength{\fboxrule}{4pt}
	\fcolorbox{red}{white}{
		\begin{minipage}[t]{
			%\textwidth}\textbf{Achtung!} Aufgrund der aktuellen Lage bezüglich COVID-19 können sich die Vorlesungstermine für dieses Semester noch ändern. Schau am besten auf Alma (\url{https://alma.uni-tuebingen.de/}), ob die Termine dort geupdatet wurden.
			\textwidth}\textbf{Achtung!} Die Daten für die Vorlesungstermine können sich noch ändern. Schau am besten auf Alma (\url{https://alma.uni-tuebingen.de/}), ob die Termine dort geupdatet wurden.
		\end{minipage}}}

\begin{center} 
\footnotesize
\begin{tabular}{|c|c|c|c|c|} \hline
Zeit     & 		Montag 								&		Dienstag						&		Mittwoch		&		Donnerstag		\\		\hline\hline
08 - 09  & 	Mathematik I 							&										& Mathematik I			& 						\\		\cline{1-1}\cline{3-3}\cline{5-5}
09 - 10  & 	(\Matheprof)							& 										& (\Matheprof)			&  						\\		\hline
10 - 11  & Tierphysiologie: Neuro-/					& Einführung Kognitionswissenschaft$^*$	& 						& Statistik I			\\		\cline{1-1}\cline{4-4}
11 - 12  & Sinnesphysiologie (Prof. Andreas Nieder) & (Prof. Rolke) 						& 						& (Prof. Franz) 				\\		\hline
12 - 13  & Computergestützte Statistik I: Gruppe 1	&  										& 						&  						\\		\cline{1-1}\cline{3-5}
13 - 14  & (Sascha Meyen, Julian Mollenhauer)		& 										& 						& 						\\		\hline
14 - 15  & Computergestützte Statistik I: Gruppe 2	& Informatik I							& 						& Informatik I 			\\		\cline{1-1}\cline{4-4}
15 - 16  & (Sascha Meyen, Julian Mollenhauer)		& (\Infoprof)  							& 						& (\Infoprof)			\\		\hline
16 - 17  &											& 										& 						& 						\\		\hline
17 - 18  &  										& 										& 						& 						\\		\hline
\end{tabular}
%\scriptsize  N6 = Hörsaalzentrum Morgenstelle
\end{center}


Dieser Plan gilt vorläufig für das erste Semester Kognitionswissenschaft. 
%Bitte informiert euch auf ALMA über Änderungen des Stundenplans aufgrund der derzeitigen Situation.
%(Die Veranstaltungen der Psychologie beginnen erst in der 2. Semesterwoche).
Es kommen noch jeweils Übungen zu den Vorlesungen Mathe\,I und Informatik\,I %und ein optionales Tutorium für Forschungsmethoden der Psychologie 
dazu.
%und optionale Tutorien für die Psychologievorlesungen dazu. 
Außerdem gibt es ein freiwilliges Tutorium zur Vorlesung Tierphysiologie: Neurobiologie und Sinnesphysiologie. %(Mo: 16-18 Uhr).

Die Zeiten für die Übungsgruppen werden innerhalb der ersten Woche in den Vorlesungen bekannt gegeben.
% Zusätzlich kann noch die Veranstaltung \emph{Introduction to Linguistics} aus dem 3. Semester vorgezogen werden (jeweils Di, Do und Fr von 8 bis 10 Uhr).
\\ \\