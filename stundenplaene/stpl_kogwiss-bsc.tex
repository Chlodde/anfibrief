Wie Ihr dem folgenden Stundenplan entnehmen könnt, enthält das Studium der Kognitionswissenschaft
im ersten Semester neben Veranstaltungen der Psychologie und Neurobiologie eine gute Portion Informatik und Mathe. 

\begin{center}
\footnotesize
\begin{tabular}{|c|c|c|c|c|} \hline
Zeit      & 			Montag 		& Dienstag							& Mittwoch 			& Donnerstag  \\
\hline\hline
08 -- 09  & 		Mathematik I, 	&									& Mathematik I, 	& \\
\cline{1-1}\cline{3-3}\cline{5-5}
09 -- 10  & 	(\Matheprof), N7	 & 									& (\Matheprof), N6 	&  \\
\cline{1-1}\cline{2-5}
10 -- 11 & Neurobio\,u.\,Sinnesphy- & Kognitions-						& 					& Statistik I \\
\cline{1-1}\cline{4-4}
11 -- 12 & siologie (Mallot), N1	& wissenschaft 1\,(Rolke),\,H4329	& 					& (Franz), H7 \\
\hline
12 -- 13 & & & & \\
\hline
13 -- 14 & & & & \\
\hline %\cline{1-2}\cline{3-6}
14 -- 15 &  						& Informatik I 						&					 & Informatik I  \\
\cline{1-2}\cline{4-4}
15 -- 16 &  						&  (\Infoprof), N6  				&					& (\Infoprof), N6  \\
\hline
16 -- 17 & & & &\\
\hline
17 -- 18 & & & & \\
\hline
\end{tabular}\\
\scriptsize  N1, N6, N7 = Hörsaalzentrum Morgenstelle; N12 = E-Bau Morgenstelle (E3A07); H4329 = Psychologisches Institut \\
F119 = Sand 6, kleiner Hörsaal
\end{center}

Dieser Plan gilt für das erste Semester Kognitionswissenschaft.
%(Die Veranstaltungen der Psychologie beginnen erst in der 2. Semesterwoche).
Es kommen noch jeweils Übungen zu den Vorlesungen Mathe\,I und Informatik\,I %und ein optionales Tutorium für Forschungsmethoden der Psychologie 
dazu.
%und optionale Tutorien für die Psychologievorlesungen dazu. 
Außerdem gibt es ein Seminar zur Vorlesung Neurobiologie und Sinnesphysiologie (Mo 14-16 Uhr oder Do 16-18 Uhr).

Die Zeiten für die Übungsgruppen werden innerhalb der ersten Woche in den Vorlesungen bekannt gegeben.
% Zusätzlich kann noch die Veranstaltung \emph{Introduction to Linguistics} aus dem 3. Semester vorgezogen werden (jeweils Di, Do und Fr von 8 bis 10 Uhr).
