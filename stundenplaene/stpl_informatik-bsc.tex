Wie Ihr dem folgenden Stundenplan entnehmen könnt, enthält das Informatik-Studium im ersten
Semester neben der Informatikvorlesung auch eine gute Portion Mathe. Beachtet, dass Ihr
%auch ein Nebenfach ("`Schwerpunktbereich"') wählen und 
in diesem ersten Semester auch 6LP\footnote{Jede bestandene Veranstaltung wird mit einer bestimmten Anzahl an \emph{Leistungspunkten} belohnt. Diese Punkte werden auch auch Credit Points oder ECTS-Punkte genannt} an allgemeinbildenden Veranstaltungen ("'Studium Professionale"' oder früher "`überfachliche berufsfeldorientierte Kompetenzen (ÜbK)"') belegen solltet.

\begin{center}
	\begin{tabular}{|c|c|c|c|c|c|} \hline
		Zeit      & 			Montag 		& Dienstag			& Mittwoch 			& Donnerstag 			& Freitag	 \\
		\hline\hline
		08 -- 09  & 		Mathematik I 	&  					& Mathematik I 		&  						&			\\
		\cline{1-1}\cline{3-3}\cline{5-6}
		09 -- 10  & 		(\Matheprof), N7& 					& (\Matheprof), N6  &  						&			\\
		\hline
		10 -- 11  &	Einführung in die TI	&					&					&						&			\\
		\cline{1-1}\cline{3-6}
		11 -- 12 & 	(Bringmann), N7			&  					&					&			 			& 			\\
		\hline
		12 -- 13 & 							& 				 	& 				    & 						& 			 \\
		\hline
		13 -- 14 & 							& 					& Einf. in die TI, N5	& 						& 			 \\
		\hline
		14 -- 15 & 							& Informatik I 		& 					& Informatik I 			& 				\\
		\cline{1-2}\cline{4-4}\cline{6-6}
		15 -- 16 &							 & (\Infoprof), N6 	& 					& (\Infoprof), N6 		& 				\\
		\hline
		16 -- 17 & & & & &\\
		\hline
		17 -- 18 & & & & & \\
		\hline
	\end{tabular}
	
\scriptsize TI = Technische Informatik

\end{center}

Dieser Plan gilt für das erste Semester Informatik. Es kommen noch jeweils zwei Übungsstunden zu jeder Vorlesung 
sowie Veranstaltungen aus dem oben genannten "`\"UbK"'-Bereich dazu.
Die Zeiten für die Übungsgruppen werden innerhalb der ersten Woche in den Vorlesungen bekannt gegeben.
