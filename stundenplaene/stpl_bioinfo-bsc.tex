Wie Ihr dem folgenden Stundenplan entnehmen könnt, enthält das Bioinformatik-Studium im ersten
Semester neben den Informatik- und Biologie-Vorlesungen auch eine gute Portion Mathe und Chemie (Termin für Anorganische Chemie folgt, haltet euch diesbezüglich einfach über Alma auf dem Laufenden (selber Link wie unten). Falls ihr unter Bioinformatik Chemie I nichts findet, sucht einfach nach Chemie für Naturwissenschaftler.).
Das ist zwar etwas anstrengend, aber mit der richtigen Planung und unseren Tipps ist es gut machbar.\\
\noindent\makebox[\textwidth][c]{%
	\setlength{\fboxrule}{4pt}
		\fcolorbox{red}{white}{
				\begin{minipage}[t]{
					\textwidth}\textbf{Achtung!} Aufgrund der aktuellen Lage bezüglich COVID-19 können sich die Vorlesungstermine für dieses Semester noch ändern. Termine für Vorlesungen, die mit einem $^*$ markiert sind, sind unter Vorbehalt. Schaut am besten auf Alma (\url{https://alma.uni-tuebingen.de/alma/pages/cm/exa/coursecatalog/showCourseCatalog.xhtml?_flowId=showCourseCatalog-flow&_flowExecutionKey=e1s1}), ob die Termine dort geupdatet wurden.
				\end{minipage}}}

\begin{minipage}{\textwidth}
    \footnotesize
\begin{center}
\begin{tabular}{|c|c|c|c|c|c|} \hline
Zeit     & Montag 		& Dienstag		& Mittwoch 		& Donnerstag 				& Freitag\\\hline\hline
08 -- 09 & Mathematik I &  				& Mathematik I 	&  							&\\\cline{1-1}\cline{3-3}\cline{5-6}
09 -- 10 & (\Matheprof) &   			& (\Matheprof)  &  							&\\\hline
10 -- 11 &				&				&				&							&\\\hline
11 -- 12 & 				&  				&				& Organische Chemie für Nat.&\\\cline{1-4} \cline{6-6}
12 -- 13 & Biomoleküle	& Biomoleküle   & Biomoleküle   & (Jun.-Prof. Fleischer)	&\\\cline{1-1}\cline{5-6}
13 -- 14 & und Zelle    & und Zelle 	& und Zelle     & Biomoleküle und Zelle		&\\\hline
14 -- 15 & 				& Informatik I  &               & Informatik I 				&\\\cline{1-2}\cline{4-4}\cline{6-6}
15 -- 16 &			    & (\Infoprof) 	& 				& (\Infoprof)				&\\\hline
16 -- 17 & & & & &\\\hline
17 -- 18 & & & & & \\\hline
\end{tabular}
\end{center}
\end{minipage}
Dieser Plan gilt für das erste Semester Bioinformatik. Es kommen noch einige Übungsstunden
zu den einzelnen Vorlesungen dazu. Die Zeiten für die Übungsgruppen werden innerhalb der ersten Woche in den Vorlesungen bekannt gegeben.\\
%Beachtet bitte, dass die Vorlesung "`Anorganische Chemie"' erst in der \textbf{zweiten Vorlesungswoche} beginnt.\\
Allgemeine/anorganische Chemie und organische Chemie bilden zusammen Teil A des Moduls ''Chemie I''. (Die Vorlesung Allgemeine/anorganische Chemie wird noch auf Alma eingetragen.)
Für BMZ (Biomoleküle und Zelle) müsst ihr eine Übungsgruppe belegen, der genaue Ablauf wird in der Vorlesung geklärt.
%BMZ (Biomoleküle und Zelle) wird bis Ende November im Block unterrichtet, ab der zweiten Semesterwoche kommt noch ein Praktikum hinzu.
%, ab Mitte Januar dann Tierphysiologie. In den Semesterferien wird noch ein Tierphysiologisches Praktikum hinzu kommen.


