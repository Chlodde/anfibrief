Wie Ihr dem folgenden Stundenplan entnehmen könnt, enthält das Bioinformatik-Studium im ersten
Semester neben den Informatik- und Biologie-Vorlesungen auch eine gute Portion Mathe und Chemie (Termin für Chemie folgt).
Das ist zwar etwas anstrengend, aber mit der richtigen Planung und unseren Tipps ist es gut
machbar.\\
\fcolorbox{red}{white}{
		\begin{minipage}[t]{
			\textwidth}\textbf{Achtung!} Aufgrund der aktuellen Lage bezüglich COVID-19 können sich die Vorlesungstermine für dieses Semester noch ändern. Termine für Vorlesungen, die mit einem $^*$ markiert sind, sind unter Vorbehalt. Schaut am besten auf Alma (\url{https://alma.uni-tuebingen.de/alma/pages/cm/exa/coursecatalog/showCourseCatalog.xhtml?_flowId=showCourseCatalog-flow&_flowExecutionKey=e1s1}), ob die Termine dort geupdatet wurden.
		\end{minipage}}

\begin{minipage}{\textwidth}
    \footnotesize
%\begin{center}
\begin{tabular}{|c|c|c|c|c|c|} \hline
Zeit      & 			Montag 		& Dienstag			& Mittwoch 			& Donnerstag 			& Freitag	 \\\hline\hline
08 -- 09  & 		Mathematik I 	&  					& Mathematik I 		&  						&			\\\cline{1-1}\cline{3-3}\cline{5-6}
09 -- 10  & 		(\Matheprof)& 					    & (\Matheprof)      &  						&			\\\hline
10 -- 11  &							&					&					&						&			\\\hline
11 -- 12 & 							&  					&					&  			            & 			\\\hline
12 -- 13 & 			Biomoleküle$^2$	&  	                & Biomoleküle$^2$    & 			            &  \\\cline{1-1}\cline{3-3}\cline{5-6}
13 -- 14 & 				und Zelle   & 		            & und Zelle         & 						&  \\\hline
14 -- 15 & 				            & Informatik I$^{*1}$ & 				&  			& 				\\\cline{1-2}\cline{4-6}
15 -- 16 &			                & (\Infoprof) 	    & 					& 		& 				\\\cline{1-3} \cline{5-6}
16 -- 17 & & & & &\\\hline
17 -- 18 & & & & & \\\hline
\end{tabular}
%\end{center}
\end{minipage}
\\
1: Vorlesung findet dieses Semester asynchron, also online in Videoformat statt. (Link folgt) Übungen werden sowohl online als auch präsent angeboten.\\
2: Vorlesung findet dieses Semester asynchron, also online in Videoformat statt. (Link folgt)
\\
Dieser Plan gilt für das erste Semester Bioinformatik. Es kommen noch einige Übungsstunden
zu den einzelnen Vorlesungen dazu. Die Zeiten für die Übungsgruppen werden innerhalb der ersten Woche in den Vorlesungen bekannt gegeben.\\
Beachtet bitte, dass die Vorlesung "`Anorganische Chemie"' erst in der \textbf{zweiten Vorlesungswoche} beginnt.
Allgemeine/anorganische Chemie bei Prof. Sirsch und organische Chemie bei Prof. Speiser bilden zusammen Teil A des Moduls ''Chemie I''.
BMZ (Biomoleküle und Zelle) wird bis Ende November im Block unterrichtet, ab der zweiten Semesterwoche kommt noch ein Praktikum hinzu.
%, ab Mitte Januar dann Tierphysiologie. In den Semesterferien wird noch ein Tierphysiologisches Praktikum hinzu kommen.

