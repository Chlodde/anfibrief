Wie Ihr dem folgenden Stundenplan entnehmen könnt, enthält das Bioinformatik-Studium im ersten
Semester neben den Informatik- und Biologie-Vorlesungen auch eine gute Portion Mathe und Chemie.
Das ist zwar etwas anstrengend, aber mit der richtigen Planung und unseren Tipps ist es gut
machbar.


\begin{center}
\begin{tabular}{|c|c|c|c|c|c|} \hline
Zeit      & 			Montag 		& Dienstag			& Mittwoch 			& Donnerstag 			& Freitag	 \\
\hline\hline
08 -- 09  & 		Mathematik I 	&  					& Mathematik I 		&  						&			\\
\cline{1-1}\cline{3-3}\cline{5-6}
09 -- 10  & 		(\Matheprof), N7& 					& (\Matheprof), N6  &  						&			\\
\hline
10 -- 11  &							&					&					&						&			\\
\hline
11 -- 12 & 							&  					&					& Org. Chemie 			& 			\\
\cline{1-4}\cline{6-6}
12 -- 13 & 			Biomoleküle		& Biomoleküle   	& Biomoleküle	    & (Speiser), N7			& Biomoleküle \\
\cline{1-1}\cline{5-5}
13 -- 14 & 			und Zelle, N6	& und Zelle, N6		& und Zelle, N6		& 						& und Zelle, N6 \\
\hline
14 -- 15 & 		Anorg. Chemie 		& Informatik I 		& 					& Informatik I 			& 				\\
\cline{1-1}\cline{4-4}\cline{6-6}
15 -- 16 &			 (Sirsch), N6  & (\Infoprof), N6 	& 					& (\Infoprof), N6 		& 				\\
\hline
16 -- 17 & & & & &\\
\hline
17 -- 18 & & & & & \\
\hline
\end{tabular}
\end{center}

Dieser Plan gilt für das erste Semester Bioinformatik. Es kommen noch einige Übungsstunden
zu den einzelnen Vorlesungen dazu. Die Zeiten für die Übungsgruppen werden innerhalb der ersten Woche in den Vorlesungen bekannt gegeben.\\
Beachtet bitte, dass die Vorlesung "`Anorganische Chemie"' erst in der \textbf{zweiten Vorlesungswoche} beginnt.
Allgemeine/anorganische Chemie bei Prof. Sirsch und organische Chemie bei Prof. Speiser bilden zusammen Teil A des Moduls ''Chemie I''.
BMZ (Biomoleküle und Zelle) wird bis Ende November im Block unterrichtet, ab der zweiten Semesterwoche kommt noch ein Praktikum hinzu.
%, ab Mitte Januar dann Tierphysiologie. In den Semesterferien wird noch ein Tierphysiologisches Praktikum hinzu kommen.

