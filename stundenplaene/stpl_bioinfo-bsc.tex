Wie du dem folgenden Stundenplan entnehmen kannst, enthält das Bioinformatik-Studium im ersten
Semester neben den Informatik- und Biologie-Vorlesungen auch eine gute Portion Mathe und Chemie. \\ % (Termin für Anorganische Chemie folgt, halte dich diesbezüglich einfach über alma auf dem Laufenden. Falls du unter Bioinformatik Chemie I nichts findest, such einfach nach Chemie für Naturwissenschaftler.).\\
\noindent\makebox[\textwidth][c]{%
	\setlength{\fboxrule}{4pt}
		\fcolorbox{red}{white}{
				\begin{minipage}[t]{
					%\textwidth}\textbf{Achtung!} Aufgrund der aktuellen Lage bezüglich COVID-19 können sich die Vorlesungstermine für dieses Semester noch ändern. Schau am besten auf Alma (\url{https://alma.uni-tuebingen.de}), ob die Termine dort geupdatet wurden.
					\textwidth}\textbf{Achtung!} Die Daten für die Vorlesungstermine können sich noch ändern. Schau am besten auf Alma (\url{https://alma.uni-tuebingen.de}), ob die Termine dort geupdatet wurden.
				\end{minipage}}}

\begin{minipage}{\textwidth}
    \footnotesize
\begin{center}
\begin{tabular}{|c|c|c|c|c|c|} \hline
Zeit     & Montag 					& Dienstag		& Mittwoch 		& Donnerstag 			& Freitag		\\\hline\hline
08 -- 09 & Mathematik I 			&  				& Mathematik I 	&  						&				\\\cline{1-1}\cline{3-3}\cline{5-6}
09 -- 10 & (\Matheprof) 			&   			& (\Matheprof)  & 						&				\\\hline
10 -- 11 &							&				&				&						&				\\\hline
11 -- 12 & 							&  				&				& Organ. Chemie für Nat.&				\\\cline{1-4} \cline{6-6}
12 -- 13 & Biomoleküle				& Biomoleküle   & Biomoleküle   & (Prof. Fleischer)		& Biomoleküle	\\\cline{1-1}\cline{5-5}
13 -- 14 & und Zelle    			& und Zelle 	& und Zelle     & 						& und Zelle		\\\hline
14 -- 15 & Anorg. Chemie für Nat. 	& Informatik I  &               & Informatik I 			&				\\\cline{1-1}\cline{4-4}\cline{6-6}
15 -- 16 & (Prof. Meyer)			& (\Infoprof) 	& 				& (\Infoprof)			&				\\\hline
16 -- 17 & 							& 				& 				&						&				\\\hline
17 -- 18 & 							& 				& 				& 						&				\\\hline
\end{tabular}
\end{center}
\end{minipage}

Dieser Plan gilt für das 1. Semester Bioinformatik. Es kommen noch Übungen
zu den einzelnen Vorlesungen dazu. Die Zeiten für die Übungsgruppen werden innerhalb der ersten Woche in den Vorlesungen bekannt gegeben.
Beachte bitte, dass die Vorlesung "`Anorganische Chemie"' erst in der \textit{zweiten Vorlesungswoche} beginnt.
Allgemeine/anorganische Chemie und organische Chemie bilden zusammen Teil A des Moduls "`Chemie I"'.
Für BMZ (Biomoleküle und Zelle) musst du eine Übungsgruppe belegen, der genaue Ablauf wird in der Vorlesung geklärt.
%BMZ (Biomoleküle und Zelle) wird bis Ende November im Block unterrichtet, ab der zweiten Semesterwoche kommt noch ein Praktikum hinzu.
%, ab Mitte Januar dann Tierphysiologie. In den Semesterferien wird noch ein Tierphysiologisches Praktikum hinzu kommen.


