\fett{Mailinglisten}
Um euch nerviges Gerenne zu ersparen, anderen Studis unkompliziert Fragen stellen zu können und
sich zum Studium besser austauschen zu können, hat die Fachschaft Mailinglisten eingerichtet. Um
euch anzumelden schreibt einfach eine leere E-Mail an die Adresse {\ttfamily info-studium-subscribe\At fsi.uni-tuebingen.de}.
Ihr seid somit auf der Mailingliste \texttt{info-studium\At fsi.\-uni-tuebingen.de} angemeldet, auf der sich noch
viele andere Studierende befinden.
Daneben gibt es noch die Mailingliste \texttt{info-talk} für Themen, die nicht direkt mit dem Studium
zu tun haben, sowie  \texttt{info-jobs} für Stellen- und Praktikumsangebote.
Falls ihr zu den *-Informatikern gehört, die entgegen dem Klischee gerne Sport treiben, sei euch die Liste \texttt{sport} ans Herz gelegt.
\ifkogwiss 
Eine Mailingliste für Themen, die ausschließlich Kognitionswissenschaftler betreffen, ist
\texttt{kogwiss}, die Anmeldung erfolgt per Mail an \texttt{kogwiss-subscribe\At fsi.uni-tuebingen.de}. 
Die neu eingerichtete Mailingliste für die Teilnahme an Studien und Versuchen (nützlich für Versuchspersonenstunden) ist \texttt{versuche}, 
die Anmeldung erfolgt mit einer Mail an \texttt{versuche-subscribe\At fsi.uni-tuebingen.de}.
%Informationen von der Fachschaft Psychologie, sowie von Psychologiestudenten im Grundstudium
%bekommt ihr über die Mailingliste \texttt{psycho-gs}. Anmelden geht auch hier mit einer
%(leeren) Mail an \texttt{psycho-gs-subscribe\At fsi.uni-tuebingen.de}.
\fi
Zu diesen Mailinglisten meldet ihr euch ebenfalls mit einer leeren Mail an: {\ttfamily <Maillinglistenname>-subscribe\At fsi.uni-tuebingen.de}
Falls ihr euch wieder abmelden wollt: Eine leere Mail an: 
{\ttfamily <Maillinglistenname>-unsubscribe\At fsi.uni-tuebingen.de}
%Weitere Informationen dazu gibt es dann auf unseren Erstsemesterveranstaltungen.
