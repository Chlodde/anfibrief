\fett{Mailinglisten}
Um euch nerviges Gerenne zu ersparen, anderen Studis unkompliziert Fragen stellen zu können und
sich zum Studium besser austauschen zu können, hat die Fachschaft als zentrale Anlaufstelle (und um dem Chaos von diversen Facebook- und WhatsApp-Gruppen zu entgehen) Mailinglisten eingerichtet. 
Um euch auf den Mailinglisten an- und abzumelden, könnt ihr die Verwaltungsoberfläche verwenden, die ihr unter der jeweils angegebenen URL erreichen könnt. Ihr erhaltet nach der Angabe eurer Mailadresse eine Bestätigungsmail mit einem Link, den ihr aufrufen müsst. Ihr seid danach auf der Mailingliste registriert, könnt die Mails auf dieser Liste mitlesen und selbst Nachrichten an alle Mitglieder schreiben.
\begin{itemize}
\item Die wichtigste Mailadresse, unter der ihr andere Studierende erreichen könnt und unter der wir und auch unsere Professorenschaft euch wichtige Nachrichten schreiben können, ist die Mailingliste \texttt{info-studium}. \textbf{Meldet euch unbedingt auf dieser Liste an!}\\
Ihr erreicht die Oberfläche zur Anmeldung unter \url{https://www.fsi.uni-tuebingen.de/mailman/listinfo/info-studium}.
\item Daneben gibt es noch die Mailingliste \texttt{info-talk} für Themen, die nicht direkt mit dem Studium
zu tun haben, aber dennoch von Interesse sein können. Die Oberfläche zur Anmeldung befindet sich unter \url{https://www.fsi.uni-tuebingen.de/mailman/listinfo/info-talk}.
\item Für Angebote von Hiwi\footnote{wissenschaftliche/studentische Hilfskraft}-Jobs, Praktika sowie Angebote für Werksstudentenjobs etc. gibt es die Mailingliste \texttt{info-jobs.} Die Oberfläche zur Anmeldung findet ihr unter \url{https://www.fsi.uni-tuebingen.de/mailman/listinfo/info-jobs}.
\item Falls ihr zu den *-Informatikern gehört, die entgegen dem Klischee gerne Sport treiben, sei euch die Liste \texttt{sport} ans Herz gelegt. Die Oberfläche zur Anmeldung ist \url{https://www.fsi.uni-tuebingen.de/mailman/listinfo/sport}. 
\ifkogwiss 
\item Eine Mailingliste für Themen, die ausschließlich Kognitionswissenschaftler betreffen, ist
\texttt{kogwiss}, die Anmeldung erfolgt über die Verwaltungsoberfläche unter \url{https://www.fsi.uni-tuebingen.de/mailman/listinfo/kogwiss}.
\item Um die Suche nach Teilnehmern für Studien und Versuche (nützlich für Versuchspersonenstunden\footnote{oft abgekürzt mit VP}) zu vereinfachen, gibt es die Liste \texttt{versuche}, 
die Anmeldung erfolgt über die Oberfläche unter \url{https://www.fsi.uni-tuebingen.de/mailman/listinfo/versuche}. 
%Informationen von der Fachschaft Psychologie, sowie von Psychologiestudenten im Grundstudium
%bekommt ihr über die Mailingliste \texttt{psycho-gs}. Anmelden geht auch hier mit einer
%(leeren) Mail an \texttt{psycho-gs-subscribe\At fsi.uni-tuebingen.de}.
\fi
\end{itemize}
%Ihr könnt euch alternativ mit einer leeren Mail an \texttt{<Maillinglistenname>-subscribe\At fsi.uni-tuebingen.de} an einer Liste anmelden und euch mit einer leeren Mail an \texttt{<Maillinglistenname>-unsubscribe\At fsi.uni-tuebingen.de} wieder von der entsprechenden Liste abmelden. Wesentlich komfortabler geht es aber über die oben genannten URLs.
%Weitere Informationen dazu gibt es dann auf unseren Erstsemesterveranstaltungen.
