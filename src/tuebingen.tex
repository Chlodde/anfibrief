
\fett{Verkehr in Tübingen}
Zur Morgenstelle (siehe Stadtplan auf der letzten Seite) kommt ihr am besten mit den Linien 5, 13, 14, 17, 18, X15 und 19 des Tübinger
Stadtbusses (Ziel-Haltestelle BG-Unfallklinik\footnote{\textbf{nicht} "`Auf der Morgenstelle"'!}). Es gibt zwar auch Parkplätze auf der Morgenstelle,
für deren Benutzung muss man jedoch eine Freischaltung des Ausweises beantragen, außerdem ist die Anzahl an Parkplätzen sehr begrenzt und man bekommt hier oft ab ca. 10 Uhr keinen Platz mehr.
\ifmaster
Die meisten Master-Veranstaltungen finden auf dem Sand, dem Sitz des WSI, statt. Den Sand erreichst du mit der Linie 2, Haltestelle \emph{Sand Drosselweg}. Dann ca. 200 Meter weiter Richtung Süden und am Ende der Straße durch den Torbogen. Hinter dem Hauptgebäude Sand 14 befindet sich ein großer Parkplatz, der Sand ist also auch mit dem Auto gut zu erreichen. Die großen Hörsäle befinden sich im Gebäude Sand 6, die Seminarräume im Hauptgebäude Sand 1 bzw. Sand 14.
\fi
Mit Eurem Semesterbeitrag habt ihr (neben dem Studentenwerksbeitrag und der Immatrikulationsgebühr) schon einen Teil des Semester-Tickets für den Bus bezahlt.
Das Ticket kostet \ticketpreis~EUR und gilt im ganzen Naldogebiet, einem Verkehrsverbund rund um
Tübingen (leider nicht bis Stuttgart, dafür bis Überlingen am Bodensee). Das Ticket gilt ab 1. Oktober und ist u. a. beim Verkehrsverein
(Neckarbrücke), in den Reisezentren der DB und online\footnote{\url{https://tickets.naldo.de/}} erhältlich. Um das Semesterticket offline kaufen zu können, müsst ihr euren Studentenausweis sowie die "`Bescheinigung für das Semesterticket"' vom Datenkontrollblatt vorzeigen. Für den Online-Kauf benötigt ihr eine Login-ID der Uni (beginnend mit \texttt{zx}).\\
Seit dem Wintersemester 2014/2015 gilt außerdem die Freizeitregelung von Naldo: Hier könnt ihr unter der Woche ab 19 Uhr und am Wochenende ganztägig Busse und Bahnen innerhalb des Naldo-Gebiets kostenlos nutzen. Dafür braucht ihr einfach nur euren Studentenausweis, auf dem sich das Naldo-Logo befindet.
Besonders interessant, falls die Eltern zu Besuch in Tübingen sind: Samstags kann man im Stadtgebiet (Tarifstufe 11) kostenlos Bus fahren\footnote{Zumindest zum Zeitpunkt des Redaktionsschlusses gilt diese Regelung noch}, egal ob man studiert oder nicht.

\fett{Wohnen in Tübingen}
Das Thema Wohnen hat in Tübingen in den letzten Jahren leider nicht an Brisanz verloren, so dass
es unter Umständen schwierig ist, eine Unterkunft zu finden. Als erste Anlaufstellen empfehlen wir
die beiden Studentenwerke: Antragsformulare für einen Wohnheimplatz gibt es entweder beim
Studentenwerk AdöR in der Wohnheimverwaltung, Fichtenweg 5, 72076 Tübingen, oder beim Studentenwerk
e.V., Rümelinstraße 8, 72070 Tübingen. Für den privaten Wohnungsmarkt sind Mittwochs- und Samstagsausgabe
des Schwäbischen Tagblatts zu empfehlen. Weiterhin ist auch die Studentische Zimmervermittlung in der Mensa
Wilhelmstraße sehr hilfreich. Im Internet haben sich \url{www.wg-gesucht.de}, \url{www.zwischenmiete.de} und
\url{www.studenten-wg.de} etabliert. Mehr gibt es auf
unserer Homepage unter \url{https://www.fsi.uni-tuebingen.de/erstsemester/faq\#wohnen}.