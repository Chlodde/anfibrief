\ifml
    \fett{Public transport in Tübingen}
    Bus schedules can be found online\footnote{\url{https://naldo.de/}} and in the Naldo app for Android and iOS.\\
    The best way to get to the "`Morgenstelle"' (see city map on the last page) is to take the bus (destination stop BG-Unfallklinik\footnote{\textbf{not} "`Auf der Morgenstelle"'!}). There is a very limited number of parking lots at the Morgenstelle, but one has to apply for permission to use them, first.
    Most Master events/lectures take place on the "`Sand"', the place of the WSI (Wilhelm-Schickard-Institut). To get to the Sand, take line 2, stop \emph{Sand Drosselweg}. It is also easily reached by car as there are multiple parking lots.
    Some Machine Learning lectures take place in the Maria-von-Linden-Straße 6. To get there, take line 3, stop \emph{Maria-von-Linden-Straße}.
    With your semester fee, you have already paid part of the semester ticket for the bus (in addition to the student union fee and the matriculation fee).
    
    The semester ticket costs \semesterticketpreis~EUR and is valid in the whole Naldo area, a public transport system around
    Tübingen (unfortunately not to Stuttgart, but to Überlingen at Lake Constance).
    
    The Deutschlandticket JugendBW (Studierende) costs \jugendticketbwpreis~EUR and is valid in the whole of Germany (not for long-distance lines, i.e. ICE, IC, EC and long-distance coach).
    In contrast to the Semesterticket, the D-Ticekt JugendBW has an age restriction, it can only be purchased by students who are under the age of 27 at the beginning of the semester.
    Both tickets are valid from
    \ifwintersemester
    October 1st.
    \fi
    \ifsommersemester
    April 1st.
    \fi

    As an alternative for older students, there is also the normal Deutschlandticket, which is valid for one month and is available for \detickettuepreis~EUR. 
    It should be noted that the paper ticket must be ordered by the 15th of the previous month at the latest, and the mobile ticket at least two days before the start of the month.

    All three tickets are available in the DB travel centres and online\footnote{\url{https://tickets.naldo.de/}}. For the online purchase, you need a login ID of the university (starting with \texttt{zx}).
    In order to buy one of these tickets offline, you will need to show your student ID and the Semester Ticket Certificate\footnote{alma > My Studies > Student Service > Requested Reports/Reports > Naldo-Bescheinigung}. For the D-Ticket JugendBW (Studierende), you must also show an official photo ID as proof of age.\\
    
    Naldo's leisure time regulations do also apply: Here you can use buses and trams within the Naldo area free of charge on weekdays from 7 p.m. and all day on weekends. All you need is your student card with the Naldo logo on it.
    Especially interesting if your parents are visiting Tübingen: On Saturdays, you can take a free bus in the city area (tariff level 11), no matter if you are studying or not.\\ \\

    \fett{Housing in Tübingen}
    Unfortunately, the topic of housing has not lost its explosive power in Tübingen in recent years,
    so it may be difficult to find accommodation. We recommend the following two Studierendenwerke as your first points of contact:
    Application forms for a place in a hall of residence can be obtained from either the Studierendenwerk AdöR in the administration of the dormitory,
    Fichtenweg 5, 72076 Tübingen, or at the Studierendenwerk e.V., Rümelinstraße 8, 72070 Tübingen. For the private housing market, Wednesday and Saturday
    editions of the Schwäbisches Tagblatt are available.
    On the Internet the platform \url{www.wg-gesucht.de} is your best option.
    You can find more information on our homepage at \url{https://www.fsi.uni-tuebingen.de/erstsemester-faq/}.

\else

    \fett{Verkehr in Tübingen}
    Die Fahrpläne für den ÖPNV  findest du online\footnote{\url{https://naldo.de/}} und in der Naldo-App für Android und iOS.\\
    Die Bushaltestelle für die Unigebäude an der \emph{Morgenstelle} (siehe Stadtplan auf der letzten Seite) heißt "`BG-Unfallklinik"'\footnote{\textbf{nicht} "`Auf der Morgenstelle"'!}). Es gibt zwar auch Parkplätze dort,
    jedoch muss man deren Benutzung beantragen und die Anzahl an Parkplätzen ist ohnehin sehr begrenzt.
    \ifmaster
    Die meisten Master-Veranstaltungen finden auf dem \emph{Sand}, dem Sitz des WSI (Wilhelm-Schickard-Institut), statt. Die Bushaltestelle heißt "`Sand Drosselweg"'.
    \fi
    
    Das Semesterticket kostet \semesterticketpreis~EUR und gilt im ganzen Naldogebiet, dem Verkehrsverbund rund um Tübingen (leider nicht bis Stuttgart, dafür bis Überlingen am Bodensee).
    
    Das Deutschlandticket JugendBW (Studierende) kostet \jugendticketbwpreis~EUR und gilt im ÖPNV in ganz Deutschland (nicht im Fernverkehr, also ICE, IC, EC und Fernbus). Im Gegensatz zum Semesterticket gibt es beim D-Ticket JugendBW (Studierende) eine Altersbeschränkung,
    es kann nur von Studierenden gekauft werden, die zu Beginn des Semesters unter 27 Jahre alt sind.
    Beide Tickets gelten jeweils ab dem
    \ifwintersemester
    1. Oktober.
    \fi
    \ifsommersemester
    1. April.
    \fi

    Alternativ gibt es für ältere Studierende noch das Deutschlandticket, das jeweils für einen Monat gültig ist und für \detickettuepreis~EUR erhältlich ist. 
    Es ist zu beachten, dass hier das Papierticket bereits bis spätestens zum 15. des Vormonats, das Handyticket bis zwei Tage vor Monatsbeginn bestellt werden muss.

    Alle drei Tickets sind in den Reisezentren der DB und online\footnote{\url{https://tickets.naldo.de/}} erhältlich. Für den Online-Kauf benötigt ihr eine Login-ID der Uni (beginnend mit \texttt{zx}). Falls ihr eines dieser Tickets offline kaufen möchtet, müsst ihr euren Studiausweis
    sowie die "`Naldo-Bescheinigung"'\footnote{alma > Mein Studium > Studienservice > Bescheide/Bescheinigungen} vorzeigen. Für das D-Ticket JugendBW (Studierende) müsst ihr als Altersnachweis zusätzlich einen amtlichen Lichtbildausweis zeigen. 
    
    Außerdem gilt die Freizeitregelung von Naldo: Hier könnt ihr unter der Woche ab 19 Uhr und am Wochenende ganztägig Busse und Bahnen innerhalb des Naldo-Gebiets kostenlos nutzen. Dafür braucht ihr einfach nur euren Studiausweis, auf dem sich das Naldo-Logo befindet.
    Besonders interessant, falls die Eltern zu Besuch in Tübingen sind: Samstags kann man im Stadtgebiet (Tarifstufe 11) kostenlos Bus fahren, egal ob man studiert oder nicht. \\\\

    \fett{Wohnen in Tübingen}
    Das Thema Wohnen hat in Tübingen in den letzten Jahren leider nicht an Brisanz verloren, so dass
    es unter Umständen schwierig ist, eine Unterkunft zu finden. Als erste Anlaufstellen empfehlen wir
    die beiden Studierendenwerke: Antragsformulare für einen Wohnheimplatz gibt es entweder beim
    Studierendenwerk AdöR in der Wohnheimverwaltung, Fichtenweg 5, 72076 Tübingen, oder beim Studierendenwerk
    e.V., Rümelinstraße 8, 72070 Tübingen. Für den privaten Wohnungsmarkt sind Mittwochs- und Samstagsausgabe
    des Schwäbischen Tagblatts zu empfehlen.
    Im Internet hat sich \url{www.wg-gesucht.de} etabliert. Mehr gibt es auf unserer Homepage unter \url{https://www.fsi.uni-tuebingen.de/erstsemester-faq/}.
    \fi
