\fett{Die wichtigsten Online-Portale}
Um keine wichtigen Informationen zu verpassen, solltest du dich mit den Online-Lehrportalen der Uni Tübingen vertraut machen
\footnote{Ja, wir sprechen bewusst in der Mehrzahl, denn wo und wie Lehrmaterialien und Informationen veröffentlicht werden, ist leider immer noch
von Lehrstuhl zu Lehrstuhl unterschiedlich und man findet sich bei so manchem Modul in einer stattlichen Link-Sammlung wieder.}.
\begin{itemize}
	\item \textbf{\hypertarget{alma}{alma}} \url{https://alma.uni-tuebingen.de/} \\
	alma ist das zentrale Vorlesungsverzeichnis der Uni Tübingen. Es ist der Startpunkt, von dem aus du dich auf das tatsächlich verwendete Portal weiterleiten lässt.
	Hier findest du alle angebotenen Veranstaltungen des ausgewählten Semesters. Unter \texttt{Studienangebot > Vorlesungsverzeichnis anzeigen} kannst du dir die Vorlesungen heraussuchen. Die Prüfungsanmeldung gegen Ende des Semesters erfolgt meist auch über alma.
	\item \textbf{Ilias/moodle} \url{https://ovidius.uni-tuebingen.de/}, \url{https://moodle.zdv.uni-tuebingen.de/} \\
	Ilias und Moodle bieten verschiedene Kommunikationstools an und sind die am häufigsten verwendete Portale, um eine Vorlesung zu organisieren. Auch hier könnt ihr einzelne Veranstaltungen suchen, jedoch empfehlen wir, zuerst bei alma reinzuschauen. 
	\item Zusätzlich gibt es dann noch eigene Foren, Teaching Websites bis hin zu Discord Servern, auf die du aber allesamt über die Portale oben weitergeleitet wirst, falls sie verwendet werden.
\end{itemize}

% TODO nächstes Semester: Anpassen
\ifbachelor \pagebreak \fi

\fett{Die Anfangszeiten und Orte}
Wie du sicher bald bemerken wirst: 9 Uhr heißt 9.15 Uhr. An der Uni fangen Veranstaltungen in der Regel c.t. (\textit{cum
tempore}, lat. \glqq mit Zeit\grqq) an. Wenn etwas „pünktlich“ anfängt, wird es als s.t. (\textit{sine tempore}, lat. \glqq ohne
Zeit\grqq) angekündigt.
Bezüglich der genauen Zeiten und Orte solltest du vor Vorlesungsbeginn unbedingt Blick in das \hyperlink{alma}{alma}-Portal werfen.

% TODO nächstes Semester: Anpassen
\ifinfo \ifmaster \pagebreak \fi \fi

\ifbachelor
\fett{Die ersten Vorlesungen}
Hier ein kleiner Überblick zu den wichtigsten Vorlesungen:
\begin{itemize}
	\item
	\textbf{Die Informatik-Vorlesung} \\
	Die
	\ifwintersemester Informatik I \fi
	\ifsommersemester Informatik II \fi
	ist eine gründliche Einführung in die Entwicklung von Programmen und die dazugehörigen formalen Grundlagen.
	Du verbringst einen Großteil der Zeit damit, eigene Programme zu schreiben und wirst dabei Schritt für Schritt
	an Konzepte wie Rekursion oder Abstraktion herangeführt.
	\ifwintersemester In der Informatik I wird dazu die funktionale Programmiersprache DrRacket verwendet. \fi
	\ifsommersemester Die Vorlesung Informatik II ist unabhängig von der Informatik I und kann auch zu Beginn gehört werden. \fi
	%\fett{Die Informatik-Vorlesung}
	%Die Vorlesung Informatik II ist unabhängig von der Informatik I und bietet euch einen Einstieg
	%in ein vielen noch unbekanntes Thema, die funktionale Programmierung. Im Zentrum stehen dabei Techniken der Abstraktion von Teilproblemen, um den Ablauf komplexerer Programme im Kern zu verstehen und somit Fehler beim Programmieren besser vermeiden zu können.

	\ifinfo
	\ifwintersemester
	\item
	\textbf{Technische Informatik} \\
	In \glqq Einführung in die Technische Informatik\grqq \ lernt ihr von den physikalischen Basics bis hin zu Dioden und Transitoren alle notwendigen Grundlagen, um die Funktionsweise von Computern zu verstehen. Auch wenn die Informatik oder Mathematik scheinbar wichtiger sind, solltet ihr hier immer am Ball bleiben! Die Vorlesung basiert weitestgehend auf dem Buch \glqq Technische Informatik\grqq \ von Wolfram Schiffmann.
	\fi
	\fi

	\item
	\textbf{Die geliebte Mathematik} \\
	Die Mathe-Vorlesungen gehören zu den Herausforderungen der ersten Semester und du wirst merken, dass in der Uni-Mathematik ein sehr großer Teil an Selbststudium von dir erwartet wird. Die Vorlesungen orientieren sich an dem Lehrbuch: „Mathematik für Informatiker und Bioinformatiker“ von Prof. Hauck, Prof. Wolff und Prof. Küchlin. Eine gute Möglichkeit zur Einstimmung bietet der „Mathematische Vorbereitungskurs für das Studium der Informatik“ (siehe oben). 
	\iflehramt
	Mathematik-Studierende brauchen Mathematik I
	und II nicht besuchen, da diese Module durch Analysis und Lineare Algebra
	abgedeckt werden. (Weitere Informationen dazu finden sich in der Prüfungsordnung.).
	\fi
\end{itemize}
\fi
