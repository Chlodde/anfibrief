
\fett{Die Anfangszeiten und Orte}
Wie du sicher bald bemerken wirst: 9 Uhr heißt 9.15 Uhr. An der Uni fangen Veranstaltungen in der Regel c.t. (cum
tempore - lat., mit Zeit) an. Wenn etwas „pünktlich“ anfängt, wird es als s.t. (sine tempore - lat., ohne
Zeit) angekündigt. Die Informatik- und Mathematik-Vorlesungen finden in den Hörsälen auf der
Morgenstelle statt. Bezüglich der Zeiten und Orte solltest du vor Vorlesungsbeginn in jedem Fall noch einmal einen Blick in das Campus-System werfen:\\
\url{http://campus.verwaltung.uni-tuebingen.de/}

\ifbachelor
\fett{Die Informatik-Vorlesung}
Zitat aus der Vorlesungsbeschreibung: „Was ist ein gutes Programm? Selbstredend ist, dass es ein
Problem löst, wenn es fertig ist. Aber wie entsteht es? Was passiert, wenn das erste Problem gelöst
ist und ein neues auftaucht? Wie weit kann das Programm wachsen, ohne dass es unter dem Gewicht
seiner Komplexität zusammenbricht? 
\ifwintersemester
Informatik I 
\fi
\ifsommersemester
Informatik II
\fi 
ist eine gründliche Einführung in die Entwicklung
von Programmen und die dazugehörigen formalen Grundlagen. Im Zentrum stehen dabei Techniken der Abstraktion
für die systematische Entwicklung und Erweiterung von Programmen. Die vermittelten Grundlagen und Techniken
sind unabhängig von einer bestimmten Programmiersprache.“

%\fett{Die Informatik-Vorlesung}
%Die Vorlesung Informatik II ist unabhängig von der Informatik I und bietet euch einen Einstieg
%in ein vielen noch unbekanntes Thema, die funktionale Programmierung. Im Zentrum stehen dabei Techniken der Abstraktion von Teilproblemen, um den Ablauf komplexerer Programme im Kern zu verstehen und somit Fehler beim Programmieren besser vermeiden zu können.

\ifinfo
\ifwintersemester
\fett{Technische Informatik}
In „Einführung in die Technische Informatik“ lernt ihr von den physikalischen Basics bis hin zu Dioden und Transitoren alle notwendigen Grundlagen, um die Funktionsweise von Computern zu verstehen. Auch wenn die Informatik oder Mathematik scheinbar wichtiger sind, solltet ihr hier immer am Ball bleiben! Die Vorlesung basiert weitestgehend auf dem Buch „Technische Informatik“ von Wolfram Schiffmann.
\fi
\fi

\fett{Die geliebte Mathematik}
Die Mathe-Vorlesungen gehören zu den Herausforderungen der ersten Semester. Sie orientieren
sich an dem Lehrbuch: „Mathematik für Informatiker und Bioinformatiker“ von Prof.
Hauck, Prof. Wolff und Prof. Küchlin. Eine gute Möglichkeit zur Einstimmung bietet der „Mathematische
Vorbereitungskurs für das Studium der Informatik“ (siehe oben).
\iflehramt
Mathematik-Studenten brauchen Mathematik I
und II nicht besuchen, da diese Module durch Analysis und Lineare Algebra
abgedeckt werden. (Weitere Informationen dazu finden sich in der
Prüfungsordnung.).
\fi
\fi