
% TODO nach Corona wieder reinnehmen.
%\ifbachelor
%    \iflehramt
%        \item[Dienstag, 3. November \YEAR, 14 Uhr, online]\ \\
%            Deine erste Vorlesung beginnt. \\
%            Du hast um 14 Uhr „Informatik 1“ bei \Infoprof.
%            Alles, was du heute (und in Zukunft) benötigst: persönlichen Wachmacher, einen Stift und einen Block. %und den Studierendenausweis.
%
%            %\seticon{faBus}~\textbf{Bushaltestelle:} BG Unfallklinik (Linie 5, 13, 14, 17, 18, 19, X15)
%    \else
%        \item[Montag, 2. November \YEAR, 8 Uhr, online]\ \\
%            Deine erste Vorlesung beginnt. \\
%            Du hast um 8 Uhr „Mathematik 1“ bei \Matheprof.
%            Alles, was du heute (und in Zukunft) benötigst: persönlichen Wachmacher, einen Stift und einen Block. %und den Studierendenausweis.
%
%            %\seticon{faBus}~\textbf{Bushaltestelle:} BG Unfallklinik (Linie 5, 13, 14, 17, 18, 19, X15)
%    \fi
%\fi

%
%Online Stadtrallye
%\ifml
%    \item[Sonntag, November 1st \YEAR, 15:00, online]\ \\
%        This semester the first online city rally is planned. Come by and get to know the city and the university virtually. You will receive the link in time after registration.
%\else
%    \item[Sonntag, 1. November \YEAR, 15 Uhr, online]\ \\
%        Dieses Semester ist die erste Online Stadtrallye geplant. Komm vorbei und lerne mit deinen die Stadt und die Uni virtuell kennen. Den Link dazu bekommst du nach einer Anmeldung rechtzeitig zugeschickt.
%\fi

%Filmabend
%\ifml
%	\item[Tuesday, October 2nd, \YEAR, 19:00, Sand 14, room A104 (meeting point is signposted)]\ \\
%	On this evening still early into the semester, we would like to invite you to a cozy movie night at the Sand. This is a perfect opportunity to relax, get to know the Sand, meet the people of the student council and future co-eds while watching a movie \footnote{Due to license restrictions, we're not allowed to tell you what movie wil be shown}.

%	\seticon{faBus}~\textbf{bus stop:} route 2, route 6 "`Sand Drosselweg"' (signposted from there)
%\else
	%\item[Dienstag, 31. März \YEAR, 19:00 Uhr, Sand 1 A301]\ \\
    %Am Abend des zweiten Vorkurstages möchten wir dich zu einem gemütlichen Filmabend auf dem Sand einladen.
	%Hier hast du die Möglichkeit, bei einem Film \footnote{welcher Film gezeigt wird, dürfen wir aus lizenzrechtlichen Gründen nicht bekannt geben.} zu entspannen, einige Fachschaftler, den Sand und eure zukünftigen Kommilitonen kennen zu lernen.

	%\seticon{faBus}~\textbf{Bushaltestelle:} Linie 2, Linie 6, Sand Drosselweg (Rest ausgeschildert)
%\fi

% Bus-Schnitzeljagd, neu im WS19/20
%\ifml
%	\item[Monday, October 7th \YEAR, 19:00, \textbf{in front of} Neckarmüller]\ \\
%    During a scavenger hunt you will be sent off in teams to explore the Tübingen bus network. By solving various puzzles, you will not only get to know your new fellow students better, but also get to know some stops that you will encounter more or less frequently in your everyday university life. Since students in Tübingen (and in the complete naldo area) are allowed to take the bus and train free of charge from Monday to Friday from 19:00 \footnote{as well as all day on Saturdays, Sundays and public holidays in Baden-Württemberg} onwards, you only need your student ID card for your exploration tour.\\
%If details change, you will find further information such as time and meeting point on our website \url{https://www.fsi.uni-tuebingen.de/ersti}.\\

%\else
%	\item[Montag, 7. Oktober \YEAR, 19:00 Uhr, vor dem Neckarmüller]\ \\
%    Bei der Schnitzeljagd wirst du mit deinen Kommilitonen in Teams losgeschickt, um das Tübinger Busnetz zu erkunden. Durch das Lösen verschiedener Rätsel lernt ihr dabei nicht nur eure neuen Komilitonen sondern auch einige Haltestellen besser kennen, die euch in eurem Uni-Alltag mehr oder weniger häufig begegnen werden. Da Studenten in Tübingen (und im kompletten naldo-Bereich) montags bis freitags ab 19:00 Uhr \footnote{sowie ganztägig an Samstagen, Sonntagen und gesetzlichen Feiertagen in Baden-Württemberg} kostenlos Bus und Bahn fahren dürfen, benötigt ihr für eure Erkundungstour lediglich euren Studierendenausweis.\\
%    Falls sich Details ändern sollten, findest du weitere Infos wie Uhrzeit und Treffpunkt auf unserer Webseite \url{https://www.fsi.uni-tuebingen.de/ersti}.\\
%\fi


% Frühstück
%\ifml
% Frühstück Master ML
%    \item[Friday, October 11th \YEAR, 9:00, Mensa Morgenstelle]\ \\
%        Let's have breakfast together, we're paying! You will get to know a lot of stuff about the university, the student council, what it's like to study in Tübingen and what you can expect in the upcoming months -- especially when talking to older co-eds.
%    \ifwintersemester You will also be greeted by \Infoprof~-- he will lecture Informatics I. This is a lecture for bachelor students, but maybe you will get to know him in other lectures as well. \fi
%    Afterwards, we will provide a tour of the Morgenstelle so you get to know the most important rooms and lecture halls.
%    You can reach the Morgenstelle via bus, bicycle or on foot. However, the way up to the Morgenstelle is quite steep -- Tübingen is hilly (German: \emph{hügelig}), so you should factor in the time you need to get there.
%    The simplest (but also steepest) way is via "`Parkhaus König"' and up "`Schnarrenbergstraße"', past the university hospital. If you're starting at Waldhäuser Ost, follow the "`Nordring"' in direction of "`Uni-Morgenstelle"' / "`Kliniken Berg"'.
%    If you want to drive there by car, you can park (not free!) by the side of the road on Nordring or inside the multi-story cark park "`Ebenhalde"' north of the Morgenstelle. However, the simplest way is via bus.
%    From 11:45 onwards, you can try out the mensa food if you like.\\
%    \seticon{faBus}~\textbf{bus stop:} BG Unfallklinik (bus routes 5, 13, 14, 17, 18, 19, X15)

%\else
    %\item[Donnerstag, 9. April \YEAR, 9:00 Uhr, Sand 1 A104]\ \\
        %Wir laden dich an diesem Morgen zu einem gemütlichen Frühstück ein! Dabei erfährst du einiges über die Uni, die Fachschaft und was dich in den nächsten Monaten erwartet -- auch im Gespräch mit älteren Studierenden.
%    \ifwintersemester Außerdem wirst du durch \Infoprof~-- er wird die Informatik 1 Vorlesung halten -- begrüßt. \fi
    %\ifsommersemester Außerdem wirst du durch \Infoprof~-- er wird die Informatik 2 Vorlesung halten -- begrüßt. \fi
    %\ifmaster Zwar ist Informatik 1 eine Bachelor-Veranstaltung, aber du wirst \Infoprof~ vielleicht auch in Master-Vorlesungen kennen lernen. \fi
%    \ifwintersemester
%        Danach machen wir eine Führung über die Morgenstelle, damit du die wichtigsten Räume und Hörsäle kennen lernst.
%        Zur Morgenstelle kommt man entweder mit dem Bus, zu Fuß oder mit dem Rad. Da der Weg zur Morgenstelle aber sehr steil ist (Tübingen ist hügelig), sollte man hierfür einiges an Zeit einrechnen.
%        Der einfachste Weg ist hier über das Parkhaus König. Von dort musst du bergauf der Schnarrenbergstraße folgen. Es geht dann zunächst an den Uni-Kliniken Berg vorbei, anschließend erreichst du die Morgenstelle. Falls du aus der Richtung Waldh\"auser-Ost kommt, so musst du dem Nordring in Richtung Kliniken Berg folgen. Für beide Wege solltest du jeweils mindestens eine halbe Stunde zu Fuß einrechnen.
%        Wenn du mit dem Auto ankommst, kannst du (kostenpflichtig) an den Straßenseiten des Nordrings, oder im Parkhaus "`Ebenhalde"' oberhalb der Morgenstelle parken. Am einfachsten geht es jedoch mit dem Bus.
%        Wenn du Lust hast, kannst du ab 11:45 Uhr das Mensaessen ausprobieren.
%    \fi

%    \ifwintersemester \seticon{faBus}~\textbf{Bushaltestelle:} BG Unfallklinik (Linie 5, 13, 14, 17, 18, 19, X15) \fi
    %\ifsommersemester \seticon{faBus}~\textbf{Bushaltestelle:} Linie 2, Linie 6, Sand Drosselweg \fi
%\fi

%\iflehramt
%    \ifbachelor
%        \item[Donnerstag, 9. April \YEAR, 10:00-12:00 Uhr, Kupferbau, Hörsaal 22]\ \\
%            Ersits im \textbf{Bachelor of Education} können beim Frühstück (rein theoretisch) nicht mit dabei sein, denn parallel zum Frühstück werdert ihr um 10:00 Uhr über die speziellen Anforderungen des Lehramtsstudiums informiert. Neben den eigentlichen Fachinhalten kommen im Bachelor of Education noch einige andere Dinge auf euch zu, z.B. ein Orientierungspraktikum, die Fachdidaktik und der Studienbereich Bildungswissenschaften. Zu Beginn des Masters steht dann ein Schulpraxissemester an, und nach dem Studienabschluss das Referendariat. Bei dieser Veranstaltung werden euch alle diese Elemente des Lehramtsstudiums vorgestellt.
%
%        \seticon{faBus}~\textbf{Bushaltestelle:} Hölderlinstraße bzw. Uni/Neue Aula
%    \fi
%    \ifmaster
%        \item[Donnerstag, 9. April \YEAR, 14:00-16:00 Uhr, Kupferbau, Hörsaal 22]\ \\
%            Wenn du im \textbf{Master of Education} studierst, kannst du zum Frühstück bleiben, eure Einführungsveranstaltung beginnt dann erst um 14 Uhr. Hier bekommst du u.a. einen Einblick in die verschiedenen Studienanteile des Master of Education und Informationen zum Schulpraxissemester.
%    \fi
%\fi

%Begrüßung Online
%\item[Freitag, 30. Oktober \YEAR, 10:00 Uhr, via Zoom]
%An diesem Tag werdet ihr von den Professoren aus unserem Fachbereich begrüßt. Nähere Details folgen.


%\ifwintersemester
%    \ifml
%	    \item[Friday, October 11th, \YEAR, 14:00, Sand, Foyer (rooms and schedule will follow)]\ \\
%    	At 2 o'clock, we'll meet up again at the Sand, home of the faculty of computer science. % Ich weiß dass es keine Fakultät ist, übersetz mir mal einer Fachbereich...
%    	Here you can gain some interesting insights into the work of the different work groups present at the Sand.
%	    Getting to know the work groups makes sense for ML as well, depending on how you want to focus your studies or thesis. Depending on your interests you can participate in different mini workshops or listen to talks, however, this requires some planning on your part. The talks available (as well as the time slots) are subject to change, please refer to \url{https://www.fsi.uni-tuebingen.de/ersti} the day before.
%    	For the most part, the work groups focusing on Machine Learning are located at the Tübinen AI research building, however, access there is rather restricted.
%
%	    \seticon{faBus}~\textbf{bus stop:} route 2, route 6, "`Sand Drosselweg"' (signposted from there)
%    \else
%	    \item[Freitag, 11. Oktober \YEAR, 14 Uhr, Sand 1, Foyer (Räume und Programm folgen)]\ \\
%    	Um 14:00 treffen wir uns auf dem Sand (dem Sitz der Informatik in Tübingen) wieder. Hier werden wir dich mit Informationen rund um das Studium und die Arbeit unserer Lehrstühle auf dem Sand versorgen.
%    	Nachdem du über den Verlauf des Studiums der ersten Wochen, Monate und Semester informiert wurdest, bieten dir die verschiedenen Fachbereiche mit interaktiven Vorträgen einen Einblick in ihre Arbeit. Die Fachbereiche kennen zu lernen lohnt sich für Studierende aller Studiengänge. Die Vorträge zeigen, welche Themen am Wilhelm-Schickard-Institut verfolgt werden, geben dir ein Gefühl, was für dich hier interessant sein kann und wo du möglicherweise Schwerpunkte im Studium, einer Studien- oder Abschlussarbeit setzen willst. Je nach Interesse kannst du dir verschiedene Vorträge anhören, allerdings bedarf dies an Auswahl und Planung von deiner Seite. Die angebotenen Vorträge können sich thematisch oder, wie die Begrüßungen, im Zeitrahmen ändern. Schau daher auch kurzfristig (am selben Tag) auf \url{https://www.fsi.uni-tuebingen.de/ersti} nach.
%
%        \seticon{faBus}~\textbf{Bushaltestelle:} Linie 2, Linie 6, Sand Drosselweg (Rest ausgeschildert)
%    \fi
%\fi

%Mastercafé
%\ifmaster
%    \ifml
%        \item[Friday, October 11th \YEAR, 11:00, Maria-von-Linden-Straße 6]\ \\
%    At this informal meeting you will have the opportunity to get to know your fellow students and some of the lecturers over a coffee. And then there will be cake.\footnote{This is not a lie!}

%        \seticon{faBus}~\textbf{bus stop:} route 3, "`Maria-von-Linden-Straße"' (signposted from there)
%    \else
    %\item[Dienstag, 14. März \YEAR, 16:00 Uhr, Sand 1 A301]\ \\
    %Bei diesem informellen Treffen hast du Gelegenheit, neue Master-Studierende aus deinem eigenen und auch aus anderen Studiengängen sowie einige der Dozenten bei Kaffee kennenzulernen. Und dann gibt es Kuchen. \footnote{Das ist keine Lüge!}

        %\seticon{faBus}~\textbf{Bushaltestelle:} Linie 2, Linie 6, Sand Drosselweg (Rest ausgeschildert)
%    \fi
%\fi

%Um 13{\ifkogwiss}:45{\fi} Uhr treffen wir uns auf dem „Sand“ (dem Sitz des Wilhelm-Schickard-Instituts)
%wieder. Hier werden wir euch mit Informationen rund um{\ifkogwiss} {\else} euer Studium und {\fi}die Arbeit unserer Lehrstühle auf dem Sand versorgen.
%
%{\ifkogwiss}Als Studierende der Kognitionswissenschaften könnt ihr um 13:45 Uhr zusto{\ss}en, die anderen Studiengänge erhalten vorab eine spezifische Einführung, die für euch Montag, 15. Oktober am Psychologischen Institut stattfindet. {\else}{\ifinfo}Als Lehramt Informatik und Informatik-Studierende werden euch auch mögliche Nebenfächer vorgestellt, ihr werdet {\else}Ihr werdet als *-Informatik-Studierende {\fi}noch einmal spezifisch begrüßt und ihr erhaltet einen studiengangsspezifischen Einblick in Forschung und Lehre.{\fi}
%
%{\ifkogwiss}Die {\else}Nachdem ihr über den Verlauf des Studiums der ersten Wochen, Monate und Semester informiert wurdet, bieten euch die {\fi}verschiedenen Fachbereiche {\ifkogwiss}auf dem Sand bieten euch {\else} {\fi}mit Vorträgen einen Einblick in ihre Arbeit. Die Fachbereiche kennen zu lernen lohnt sich für Studierende aller Studiengänge, die Vorträge zeigen welche Themen am Wilhelm-Schickard-Institut verfolgt werden, geben euch ein Gefühl, was für euch hier interessant sein kann und wo ihr möglicherweise Schwerpunkte im Studium, einer Studien- oder Abschlussarbeit setzten wollt. Je nach Interesse könnt ihr euch verschiedene Vorträge anhören, allerdings bedarf dies an Auswahl und Planung von eurer Seite. Die angebotenen Vorträge können sich thematisch oder, wie die Begrüßungen, im Zeitrahmen ändern. Schaut daher auch kurzfristig (am selben Tag) auf \url{https://www.fsi.uni-tuebingen.de/ersti} nach.
%
%\seticon{faBus}~\textbf{Bushaltestelle:} Sand Drosselweg (Rest ausgeschildert)


%\item[Freitag, 12. Oktober \YEAR, 13{\ifkogwiss}:45{\fi} Uhr, Sand (Räume und Programm folgen)]\ \\
%Um 13{\ifkogwiss}:45{\fi} Uhr treffen wir uns auf dem „Sand“ (dem Sitz des Wilhelm-Schickard-Instituts)
%wieder. Hier werden wir euch mit Informationen rund um{\ifkogwiss} {\else} euer Studium und {\fi}die Arbeit unserer Lehrstühle auf dem Sand versorgen.
%
%{\ifkogwiss}Als Studierende der Kognitionswissenschaften könnt ihr um 13:45 Uhr zusto{\ss}en, die anderen Studiengänge erhalten vorab eine spezifische Einführung, die für euch Montag, 15. Oktober am Psychologischen Institut stattfindet. {\else}{\ifinfo}Als Lehramt Informatik und Informatik-Studierende werden euch auch mögliche Nebenfächer vorgestellt, ihr werdet {\else}Ihr werdet als *-Informatik-Studierende {\fi}noch einmal spezifisch begrüßt und ihr erhaltet einen studiengangsspezifischen Einblick in Forschung und Lehre.{\fi}
%
%{\ifkogwiss}Die {\else}Nachdem ihr über den Verlauf des Studiums der ersten Wochen, Monate und Semester informiert wurdet, bieten euch die {\fi}verschiedenen Fachbereiche {\ifkogwiss}auf dem Sand bieten euch {\else} {\fi}mit Vorträgen einen Einblick in ihre Arbeit. Die Fachbereiche kennen zu lernen lohnt sich für Studierende aller Studiengänge, die Vorträge zeigen welche Themen am Wilhelm-Schickard-Institut verfolgt werden, geben euch ein Gefühl, was für euch hier interessant sein kann und wo ihr möglicherweise Schwerpunkte im Studium, einer Studien- oder Abschlussarbeit setzten wollt. Je nach Interesse könnt ihr euch verschiedene Vorträge anhören, allerdings bedarf dies an Auswahl und Planung von eurer Seite. Die angebotenen Vorträge können sich thematisch oder, wie die Begrüßungen, im Zeitrahmen ändern. Schaut daher auch kurzfristig (am selben Tag) auf \url{https://www.fsi.uni-tuebingen.de/ersti} nach.
%
%\seticon{faBus}~\textbf{Bushaltestelle:} Sand Drosselweg (Rest ausgeschildert)

%Akadem. Spieleabend
%\ifml
%    \item[Thursday, October 17th, \YEAR 18:00 Sand 1, A301]\ \\
%        On this afternoon/evening we'd first like to invite you a second time to play board games with your fellow students. As soon as the time has advanced far enough, we'd like to live it up in the pubs.\\
%	\seticon{faBus}~\textbf{bus stop:} route 2, route 6 "`Sand Drosselweg"' (signposted from there)
%\else
    %\item[Donnerstag, 16. April \YEAR, 19:00 Uhr, Sand 1 A301]\ \\
%An diesem Nachmittag/Abend möchten wir dich zunächst auf den Sand einladen, um in gemütlicher Runde mit anderen Kommilitonen und Fachschaftlern Brett- und Gesellschaftsspiele zu spielen. Sobald die Zeit an diesem Abend ausreichend fortgeschritten ist, möchten wir zusammen mit dir ein wenig die Kneipen der Altstadt unsicher machen.

%\seticon{faBus}~\textbf{Bushaltestelle:} Linie 2, Linie 6, Sand Drosselweg (Rest ausgeschildert)
%\fi

%Online Spieleabend
%\ifml
%    \item[Freitag, November 6th \YEAR, 19:00, online]\ \\
%    Today the online game evening of the student council, which was introduced especially for Corona conditions, takes place. Here you meet your fellow students on Discord and play Fever.io, Drawful2.io and more! A registration is not necessary. You will find the link to the Discord server on our website in time.
%\else
%    \item[Freitag, 6. November \YEAR, 19 Uhr, online]\ \\
%        Heute findet der speziell für Corona-Bedingungen eingeführte Online-Spieleabend der Fachschaft statt. Hier kommst du auf Discord mit deinen Kommilitonen zusammen und spielst Fever.io, Drawful2.io und mehr! Eine Anmeldung ist nicht notwendig. Den Link zum Discord Server findest du rechtzeitig auf unserer Website.
%\fi

%Wanderung #2
%\ifml
%    \item[Sonntag, November 8th \YEAR, 11:00, WHO (Waldhäuserstraße 122, 72076 Tübingen)]\ \\
%    On a leisurely hike you will get to know not only your fellow students, but also a few lecturers and the worthwhile surroundings of Tübingen!\\
%    \seticon{faBus}~\textbf{bus stop:} Linie 2, 3, 4, 5 ''Ulmenweg'', WHO
%\else
%    \item[Sonntag, 08. November \YEAR, 11 Uhr, WHO (Waldhäuserstraße 122, 72076 Tübingen)]\ \\
%        Bei einer gemütlichen zweiten Wanderung lernt ihr neben euren Kommilitonen und Kommilitoninnen auch noch ein paar Dozierende und die sehenswerte Tübinger Umgebung kennen!\\
%        \seticon{faBus}~\textbf{Bushaltestelle:} Linie 2, 3, 4, 5 ''Ulmenweg'', WHO
%\fi

%\ifkogwiss
%\item[Montag, 14. Oktober \YEAR, Uhrzeit und Ort TBA]\ \\
%    Hier stellen sich die kognitionswissenschaftlichen Lehrstühle (also die verschiedenen Forschungsbereiche der Professoren) vor. Dies ist super um einen Überblick zu bekommen, was die Kognitionswissenschaft alles beinhaltet und um erste Eindrücke von den Profs zu bekommen. Auch die Fachschaft stellt sich dir hier erstmals vor. Danach gibt es noch Gelegenheit, mit der Fachschaft ein Glas Milch trinken zu gehen. Auch für Kogni-Master ist das eine super Veranstaltung.
%\fi

%\ifkogwiss
 %   \item[Mittwoch, 16. Oktober, \YEAR, Uhrzeit und Ort TBA]\ \\
  %       Damit man sich auch unter den Kognis kennenlernen kann, veranstalten wir einen Spiele- und Informationsabend für die Kogni-Erstis. Hier können Gesellschaftsspiele und bei gutem Wetter auch Tischtennis und Volleyball gespielt werden. Für Verpflegung können wir leider nicht auf eigene Kosten sorgen aber wir stellen gegen eine kleine Spende Getränke bereit und bestellen Pizza. Es werden auch einige höhersemestrige Kognis und Fachschaftler da sein, die man zum Kogni-Studium ausfragen kann. Kogni-Master sind natürlich auch herzlich eingeladen.
%	\seticon{faBus}~\textbf{Bushaltestelle:} Linie 2, Sand Drosselweg (Rest ausgeschildert)
%\fi

%Ersti-Wochenende
%\ifml
%    \item[Friday, October 18th, \YEAR, 13:00, train station]\ \\
%        If you are still motivated to meet new people after a 2 week Ersti program and the first week of university, you should spend a weekend with us in Dettingen. There you will find the "`Naturfreundehaus"' which we rented from October 18th to October 20th. On this weekend we would like to prepare you a bit more for your everyday studies and help you to get to know each other. The open slots for this weekend are limited and will be allocated quickly. If you want to participate, just send a mail to Roman (\texttt{roman.schulte\At student.uni-tuebingen.de}). For the payment of board and lodging we expect a contribution of EUR 20,- per participant. Further information will soon be available on our homepage.

%\else
%    \item[Freitag, 18. Oktober \YEAR, 13:00 Uhr, Bahnhof]\ \\
%        Falls du nach 2 Wochen Ersti-Programm und der ersten Woche Uni immer noch motiviert bist, neue Leute kennen zu lernen, dann verbringe doch mit uns ein Wochenende in Dettingen. Dort befindet sich das Naturfreundehaus, das wir vom 18.10. bis zum 20.10. gemietet haben. An diesem Wochenende möchten wir dich noch ein wenig mehr auf den Studienalltag vorbereiten und euch helfen, untereinander Kontakte zu knüpfen. Die Plätze für dieses Wochenende sind begrenzt und werden schnell vergeben. Wenn du teilnehmen möchtest, sende einfach eine Mail an Roman (\texttt{roman.schulte\At student.uni-tuebingen.de}). Um die Unterkunft und Verpflegung zu bezahlen, rechnen wir mit einem Unkostenbeitrag von \EUR{20} pro Teilnehmer. Weitere Infos findest sich in Kürze auch auf unserer Homepage.
%\fi

%Konfigabend
%\ifwintersemester
%    \ifml
%        \item[Tuesday, October 22nd, \YEAR, 19:00, Sand 1, A301]\ \\
%            On this evening we'd like to offer you the opportunity to configure your laptop computer for everyday life at the university. Also there shall be no lack of opportunity to revel in memories\footnote{https://xkcd.com/422}.\\
%            \textbf{please remember to bring the power cord to your laptop.}

%	    \seticon{faBus}~\textbf{bus stop:} route 2, "`Sand Drosselweg"' (signposted from there)
%    \else
%        \item[Dienstag, 22. Oktober \YEAR, 19:00 Uhr, Sand 1, A301]\ \\
%            An diesem Abend möchten wir dir die Gelegenheit bieten, eure Laptops für den Uni-Alltag einzurichten. Zudem soll die Gelegenheit nicht ausbleiben, in alten Erinnerungen zu schwelgen\footnote{https://xkcd.com/422}.\\
%            \textbf{Denkt bitte daran, die Stromkabel eurer Laptops mitzubringen.}

%        \seticon{faBus}~\textbf{Bushaltestelle:} Linie 2, Sand Drosselweg (Rest ausgeschildert)
%    \fi
%\fi


%\item[Freitag, 12. Oktober \YEAR, 13{\ifkogwiss}:45{\fi} Uhr, Sand (Räume und Programm folgen)]\ \\
%Um 13{\ifkogwiss}:45{\fi} Uhr treffen wir uns auf dem „Sand“ (dem Sitz des Wilhelm-Schickard-Instituts)
%wieder. Hier werden wir euch mit Informationen rund um{\ifkogwiss} {\else} euer Studium und {\fi}die Arbeit unserer Lehrstühle auf dem Sand versorgen.
%
%{\ifkogwiss}Als Studierende der Kognitionswissenschaften könnt ihr um 13:45 Uhr zusto{\ss}en, die anderen Studiengänge erhalten vorab eine spezifische Einführung, die für euch Montag, 15. Oktober am Psychologischen Institut stattfindet. {\else}{\ifinfo}Als Lehramt Informatik und Informatik-Studierende werden euch auch mögliche Nebenfächer vorgestellt, ihr werdet {\else}Ihr werdet als *-Informatik-Studierende {\fi}noch einmal spezifisch begrüßt und ihr erhaltet einen studiengangsspezifischen Einblick in Forschung und Lehre.{\fi}
%
%{\ifkogwiss}Die {\else}Nachdem ihr über den Verlauf des Studiums der ersten Wochen, Monate und Semester informiert wurdet, bieten euch die {\fi}verschiedenen Fachbereiche {\ifkogwiss}auf dem Sand bieten euch {\else} {\fi}mit Vorträgen einen Einblick in ihre Arbeit. Die Fachbereiche kennen zu lernen lohnt sich für Studierende aller Studiengänge, die Vorträge zeigen welche Themen am Wilhelm-Schickard-Institut verfolgt werden, geben euch ein Gefühl, was für euch hier interessant sein kann und wo ihr möglicherweise Schwerpunkte im Studium, einer Studien- oder Abschlussarbeit setzten wollt. Je nach Interesse könnt ihr euch verschiedene Vorträge anhören, allerdings bedarf dies an Auswahl und Planung von eurer Seite. Die angebotenen Vorträge können sich thematisch oder, wie die Begrüßungen, im Zeitrahmen ändern. Schaut daher auch kurzfristig (am selben Tag) auf \url{https://www.fsi.uni-tuebingen.de/ersti} nach.
%
%\seticon{faBus}~\textbf{Bushaltestelle:} Sand Drosselweg (Rest ausgeschildert)


%\ifkogwiss
%\ifbachelor
%\item[Montag, 15. Oktober \YEAR, 17 Uhr, Psychologisches Institut, Hörsaal]\ \\
%An diesem Abend werdet ihr von Frau Prof. Rolke und Frau Jendreyko begrüßt. Zudem stellen sich euch, nach einer kurzen Einführung in den Studienaufbau, die Lehrstühle der Kognitionswissenschaft mit ihren Themen und Forschungsgebieten vor.
%\fi
%%\ifmaster
%%\item[Montag, 15. Oktober \YEAR, 15:00 Uhr, Psychologisches Institut, Seminarraum 4.326 ]\ \\
%%Heute Abend stellen sich euch die verschiedene Lehrstühle der Kognitionswissenschaft vor, um einen Einblick in ihre Forschungsthemen zu ermöglichen.
%%\fi
%
%\seticon{faBus}~\textbf{Bushaltestelle:} Hölderlinstraße / Uni-Kliniken Tal
%\fi
%
%\ifkogwiss
%\ifmaster
%\item[Montag, 15. Oktober \YEAR, 15:00 Uhr, Psychologisches Institut, Seminarraum 4.326 ]\ \\
%Hier bekommen speziell Master-Studenten eine separate Begrüßung durch Frau Prof. Rolke, die euch einen ersten Einblick ins Studium gibt, indem sie den regulären Studienaufbau des Master Kognitionswissenschaft erläutert. %Ort und Zeit werden noch auf \url{https://www.fsi.uni-tuebingen.de/ersti} bekannt gegeben.
%\fi
%\fi
