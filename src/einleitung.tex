wir freuen uns, dass du Interesse hast, dein Studium der \studiengang in Tübingen zu beginnen.
Um dir den Einstieg in das Studium etwas zu erleichtern, bekommst du jetzt schon einmal
ein paar Informationen, die dir durch die ersten Wochen helfen sollen.
Weitere Tipps und Tricks wirst du an unseren Informationstagen %und auf dem Anfängerwochenende erhalten.
erhalten.


\ifkogwiss
\textbf{\glqq Fachschaft\grqq, wer oder was ist das eigentlich?} Wir, die Fachschaft Kognitionswissenschaft, sind einige Studierende, die sich für die Interessen der Studierenden einsetzen. Zu unserer Arbeit
gehört einerseits die Vertretung der Studierenden in den Universitätsgremien, andererseits versuchen wir, unseren
Mitstudierenden das Leben an der Universität zu erleichtern (z. B. durch Anfängerbetreuung, Feste,
Beratung, etc.). Dabei arbeiten wir eng mit der Fachschaft Informatik (fsi) zusammen.
\else
\textbf{\glqq Fachschaft\grqq, wer oder was ist das eigentlich?} Wir, die Fachschaft Informatik, sind einige Studierende aus den Fächern
Informatik, Bioinformatik, Medieninformatik, Medizininformatik und Machine Learning, die sich für die Interessen der Studierenden einsetzen. Zu unserer Arbeit
gehört einerseits die Vertretung der Studierenden in den Universitätsgremien, andererseits versuchen wir, unseren
Mitstudierenden das Leben an der Universität zu erleichtern (z. B. durch Anfängerbetreuung, Feste, Beratung, etc.). Hierbei arbeiten wir eng mit der Fachschaft Kognitionswissenschaften zusammen.
\fi

\ifbachelor 
Wir empfehlen dir, an den Anfängerveranstaltungen teilzunehmen, da die Umstellung von Schulalltag
auf Unialltag vielen oft schwer fällt. 
\fi
\ifmaster
Wir empfehlen dir, an den Anfängerveranstaltungen teilzunehmen, da die Umstellung zum Master bzw. von einer anderen Uni vielen oft schwer fällt und du so bereits einen ersten Einblick in den Alltag in Tübingen erhältst.
\fi 
Bei den Veranstaltungen werden viele ältere Studierenden
anwesend sein, die dir gerne alle Fragen beantworten. Eine Übersicht über die angebotenen
Veranstaltungen findest du auf den nächsten Seiten.

Solltest du vor (oder auch nach) unseren Infoveranstaltungen noch Fragen haben, kannst du uns
am besten per E-Mail unter \texttt{fsi\At fsi.uni-tuebingen.de} 
\ifkogwiss
sowie \texttt{kogni-fachschaft\At fsi.uni-tuebingen.de}
\fi
erreichen.
\ifkogwiss
Zusätzlich erreichst du unsere zwei Ersti-Mentorinnen Madeleine und Laura unter der E-Mailadresse \texttt{kogni-mentoren\At fsi.uni-tuebingen.de}, die dir gerne bei Fragen rund ums erste Semester zur Verfügung stehen.
\fi
Auf unserer Homepage
\url{https://www.fsi.uni-tuebingen.de} (oder \url{https://www.fs-kogni.uni-tuebingen.de/}) findest du bereits jetzt viele hilfreiche Informationen rund
ums Studium. Falls du Interesse hast, bei der Fachschaft mitzuarbeiten, komm einfach zu einer unserer Sitzungen. Den aktuellen Sitzungstermin findest du immer auf unserer Webseite. 
\ifkogwiss  Informationen zum Psychologie-Teil des Studiums und zu den Veranstaltungen der
Fachschaft Psychologie findest du unter \url{www.fs-psycho.uni-tuebingen.de}.\fi

Wir freuen uns, dich bei unseren Anfängerveranstaltungen kennen zu lernen!\\
%\vfill
\ifkogwiss
Deine Fachschaften Kognitionswissenschaft und Informatik
\else
Deine Fachschaft Informatik
\fi
\vfill

%\noindent\makebox[\textwidth][c]{%
%	\setlength{\fboxrule}{4pt}
%	\fcolorbox{green}{white}{
%		\begin{minipage}[t]{
%				\textwidth}
%			Solltest du noch nicht geimpft sein, hast du noch die Möglichkeit einen Termin über die Uni zu bekommen. Das empfehlen wir sehr, damit im Herbst möglichst viele Veranstaltungen in Präsenz statt finden können.\\
%			Weitere Hinweise findest du unter \url{https://uni-tuebingen.de/universitaet/infos-zum-coronavirus/impfen/}
%\end{minipage}}}
