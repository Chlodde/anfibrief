\fett{Mailing lists}

To save you from annoying running around, be able to ask other students questions in an uncomplicated way and exchange information about your studies, the student council has set up mailing lists as a central point of contact(and to avoid the chaos of Facebook groups as well as Whatsapp groups).
To subscribe to or unsubscribe from the mailing lists, you can use our web frontend, which you can access under the respectively stated URL. After you specified your mail address, you'll receive a confirmation mail including a link you'll have to invoke to confirm your subscription to the mailing list. After that you are registered on the mailing list and you can read mails on that list as well as write messages to all members yourself.

\begin{itemize}
    \item The most important mailing list to reach other students and under which we and our professors can send you important news is the mailing list \texttt{ info-studium}. \textbf{Subscribe to this list!}\\
You can access the web frontend to subscribe at \url{https://www.fsi.uni-tuebingen.de/mailman/listinfo/info-studium}.
\item Besides info-studium, there is the mailing list \texttt{info-talk} for topics not directly related to studies, but which might be interesting nonetheless. The web frontend to subscribe can be accessed at \url{https://www.fsi.uni-tuebingen.de/mailman/listinfo/info-talk}.
\item For job offerings to research assistant jobs, internships as well as working student jobs, there is the mailing list \texttt{info-jobs}. The web frontend to subscribe can be accessed at \url{https://www.fsi.uni-tuebingen.de/mailman/listinfo/info-jobs}.
\item In case you're part of the *-informaticians, who oppesed to clice enjoy doing sports, we strongly recommend the mailing list \texttt{sport}. The web frontend to subscribe can be accessed at \url{https://www.fsi.uni-tuebingen.de/mailman/listinfo/sport}. 
\end{itemize}
%Ihr könnt euch alternativ mit einer leeren Mail an \texttt{<Maillinglistenname>-subscribe\At fsi.uni-tuebingen.de} an einer Liste anmelden und euch mit einer leeren Mail an \texttt{<Maillinglistenname>-unsubscribe\At fsi.uni-tuebingen.de} wieder von der entsprechenden Liste abmelden. Wesentlich komfortabler geht es aber über die oben genannten URLs.
%Weitere Informationen dazu gibt es dann auf unseren Erstsemesterveranstaltungen.
