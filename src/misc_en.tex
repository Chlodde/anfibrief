
\fett{Important Online Portals}
To avoid missing out on any information concerning lecture times and places, please take a look at the two most important online learning portals.
\begin{itemize}
	\item \textbf{alma} \url{https://alma.uni-tuebingen.de} \\
	alma is the central course catalogue of the university. Once you've found your course here, you will find links to any course specific websites or portals.
	Towards the end of the semester you can register for your exams via alma as well.
	\item \textbf{Ilias/moodle} \url{https://ovidius.uni-tuebingen.de}, \url{https://moodle.zdv.uni-tuebingen.de} \\
	Ilias and moodle are the most commonly used communication tools. Although they hold lists of all courses, too, we would recommend alma as the start of your search for courses.
\end{itemize}

\fett{Starting times and locations}
Soon you'll notice: 9 o'clock means 09:15. At university lectures usually start c.t. (\textit{cum
tempore} - lat. with time). If something starts \glqq sharp\grqq, it will be announced as s.t. (\textit{sine tempore} - lat. without
time). In terms of times and locations you should take a look at alma before lectures start:\\
\url{https://alma.uni-tuebingen.de} \\ \\
