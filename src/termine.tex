\setlength{\fboxrule}{4pt}
	\fcolorbox{red}{white}{
		\begin{minipage}[t]{
            \textwidth}
                \ifml
                    \textbf{Attention!} Due to the current situation regarding COVID-19 the dates for this semester may still change. We are in constant exchange with the department and the university management and have to wait for the further course of prohibitions and guidelines.
                    Check \textbf{in any case} again and again on \url{https://www.fsi.uni-tuebingen.de/ersti}, there we will publish the latest data as soon as we know more.
                \else
                    \textbf{Achtung!} Aufgrund der aktuellen Lage bezüglich COVID-19 können sich die Anfi-Termine für dieses Semester noch ändern. Wir sind im 				ständigen Austausch mit dem Fachbereich und der Universitätsleitung und müssen den weiteren Verlauf von Verboten und Richtlinien abwarten.
                    Schaut \textbf{auf jeden Fall} immer wieder auf \url{https://www.fsi.uni-tuebingen.de/ersti} nach, dort werden wir die aktuellsten Daten veröffentlichen, sobald wir mehr wissen.
                \fi
		\end{minipage}}
\begin{description}
  % HINWEIS ZUR ANMELDUNG - GILT IMMER - NICHT LÖSCHEN
  \ifml
    \item[Note:] \textbf{\emph{Please register for all events}}, unless it is explicitly stated that no registration is necessary. Further details on the respective event and the \textbf{registration} can always be found on \url{https://www.fsi.uni-tuebingen.de/ersti}. You can participate in all events, even if they are offered twice.
  \else
    \item[Hinweis:] \textbf{\emph{Bitte melde dich zu allen Veranstaltungen an}}, außer es steht explizit dabei, dass keine Anmeldung notwendig ist. Weitere Details zur jeweiligen Veranstaltung und die \textbf{Registrierung} findest du auch immer auf \url{https://www.fsi.uni-tuebingen.de/ersti}. Du kannst an allen Veranstaltungen teilnehmen, auch wenn diese zwei mal angeboten werden.
  \fi
  % ENDE HINWEIS ZUR ANMELDUNG

\ifkogwiss
    \ifmaster
        \item[Montag, 19. Oktober \YEAR, 08:00 Uhr, Ort Psychologisches Institut 4.332 und 4.326]\ \\
        Heute beginnt der Vorbereitungskurs Mathematik speziell für Kognitionswissenschaftler im Master. Es ist nicht Pflicht daran teilzunehmen, es ist aber sehr empfehlenswert. Nicht zuletzt lernt ihr hier erste Freunde kennen! Der Vorkurs bietet euch eine Zusammenfassung des Mathestoffs, der im Bachelor Kognitionswissenschaft an der Uni Tübingen behandelt wird.
        \textbf{Wenn ihr euren Bachelor nicht an der Universität Tübingen oder in einem anderen Fach erworben habt, kann der Vorkurs für euch sinnvoll sein.} Falls ihr Quereinsteiger seid, werdet ihr dadurch an die in eurem Studium benötigten mathematischen Grundlagen herangeführt. Falls ihr bereits mathematisches Vorwissen mitbringt, ist der Kurs eine gute Gelegenheit, euer Wissen aufzufrischen.\\
         Es wäre super, wenn ihr euch mit einer kurzen Mail an \texttt{p.fischer\At student.uni-tuebingen.de} bei Paul Fischer anmelden würdet. Als Treffpunkt für Montag den 19. Oktober könnt ihr euch am Eingang des Psychologischen Instituts einfinden. Alle weiteren Infos erhaltet ihr dann per Mail.\\

%Stattfinden wird der Vorkurs im ÜR 08 in der alten Physik, das ist die Gmelinstraße 6. Diese befindet sich an der Bushaltestelle Gmelinstraße, nördlich gegenüber von der Neuen Aula. Wenn ihr auf der Wilhelmstraße seid, geht rechts an der Neuen Aula vorbei, dort findet ihr die Alte Physik an der Ecke Gmelin- und Nauklerstraße, rechts von der Neuen Aula. Seid ihr auf der Hölderlinstraße, geht links dran vorbei und die Alte Physik ist auf der linken Straßenseite. Genaue Informationen bezüglich Treffpunkt am ersten Termin erhaltet ihr dann nochmal per Mail.
%
%\seticon{faBus}~\textbf{Bushaltestelle:} Gmelinstraße, Hölderlinstraße, Uni/Neue Aula
%
%\else
    \fi
\fi

\ifml
	\item~ % Funktioniert nicht anders, don't judge me
\else
    \item[Montag, 19. Oktober \YEAR]\ \\
  Heute beginnt der Vorbereitungskurs Mathematik. Es ist nicht Pflicht daran teilzunehmen,
	es ist aber sehr empfehlenswert.
	%Nicht zuletzt lernst du hier erste Freunde kennen!
	\ifsommersemester
	Der Vorkurs bietet dir eine Wiederholung des Schulstoffes sowie eine Übersicht über den Stoff von Mathe II
	\fi
	
	\ifwintersemester
	Der Vorkurs bietet dir eine Wiederholung des Schulstoffes sowie eine Übersicht über den Stoff von Mathe I
	\fi
	und führt dich in die Terminologie ein, die du in den Mathe-Vorlesungen wiederfinden wirst.
	
	\ifmaster
	\textbf{Wenn du deinen Bachelor nicht an der Universität Tübingen oder in einem anderen Fach erworben hast, kann der Vorkurs für dich sinnvoll sein. Auf unserer Website findest du ein Skript, von dem ein Teil auch im Vorkurs besprochen wird. Damit solltest du einschätzen können, wie viel vom Stoff bereits bekannt ist und ob sich der Besuch des Vorkurses lohnt.}
	\fi
	
    Um am Vorkurs teilzunehmen, musst du dich bis zum 14. Oktober \YEAR~anmelden. Weitere Informationen erhälst du nach Anmeldung zum Kurs per E-Mail. Falls der Brief aus unerfindlichen Gründen erst nach dem 14. Oktober bei dir eintrifft oder du sonstige Fragen dazu hast, melde dich bitte bei Rüdiger Zell \texttt{ruediger.zeller@uni-tuebingen.de}\\
    \url{https://uni-tuebingen.de/de/91877}.
	% Um Anmeldung wird gebeten, sie ist aber nicht zwingend erforderlich.

	\ifsommersemester
	%Der Vorkurs findet auf dem Gelände des Wilhelm-Schickard-Instituts (oft einfach nur \emph{der Sand} genannt) statt. Du erreichst den Sand per Bus mit den Linien 2 und 6, Haltestelle Sand Drosselweg. Dann ca. 200 Meter Richtung Süden und durch den großen gelben Torbogen. Per Auto erreichst du den Sand über den Nordring (folgt den Schildern "`Uni-Sand"'), hinter dem Hauptgebäude Sand 14 befindet sich ein großer Parkplatz. Der Hörsaal F119 befindet sich jedoch nicht im Hauptgebäude, sondern im Gebäude Sand 6. Nach dem Torbogen direkt links und an den Bäumen entlang, dann stehst du vor dem Haupteingang von Sand 6. Der Hörsaal befindet sich hinter der ersten Tür links.
	%Bitte beachtet, dass das Semesterticket \emph{erst ab dem 1.4} gültig ist. Falls du davor Tübingen schon unsicher machst und Bus fahren willst, musst du dir extra Fahrscheine kaufen.
	%Natürlich kommt man zur Morgenstelle auch zu Fuß oder mit dem Rad, da es jedoch steil den Berg hochgeht, muss man hierfür einiges an Zeit einrechnen.
	\fi

	%\seticon{faBus}~\textbf{Bushaltestelle:} Linie 2, Linie 6 "`Sand Drosselweg"' (Rest ausgeschildert)
\fi

\ifmaster
    \ifbinfo
        \item[Mittwoch, 21. Oktober \YEAR, TBA, Sand]\ \\
            Heute beginnt ein Informatik-Vorkurs speziell für Bioinformatik-Studenten im Master. Dieser Vorkurs wird dringend empfohlen, wenn du aus einem fachfremden Studiengang wie z.B. Biologie oder anderen Lebenswissenschaften kommst und noch keine oder sehr wenig Erfahrung in der Informatik und der Programmierung (CLI, Java, Python, \LaTeX) hast. Der Vorkurs wird in Englisch gehalten. Anmeldeschluss ist bis zum 20. Oktober, 12 Uhr. Alle weitere Informationen und die Anmeldung findest du auf folgender Website: \\ \url{https://uni-tuebingen.de/de/91881}

        \seticon{faBus}~\textbf{Bushaltestelle:} Linie 2, Linie 6 Sand Drosselweg
    \fi
\fi

%Stadtrallye
\ifml
	\item[Saturday, October 24th \YEAR, \textasciitilde 16:00 Uhr, \footnotesize{location \& start time will be given to you after registration}]\ \\
	On this evening, you can participate in a team-based scavenger hunt across the city, where you will get to know interesting, beautiful as well as disturbing spots within Tübingen. As a side effect, you will hopefully get to know your new home town a bit better and make new friends.
	If the schedule (time and meeting point) should change somehow, you will find the new details on our website at \url{https://www.fsi.uni-tuebingen.de/ersti/}.

	%\seticon{faBus}~\textbf{bus stop:} route 2, "`Sand Drosselweg"' (signposted from there)
\else
	 \item[Samstag, 24. Oktober \YEAR, \textasciitilde 16 Uhr, \footnotesize{Ort \& Startzeit wird dir nach Anmeldung mitgeteilt}]\ \\
	 Bei der Stadtrallye lassen wir dich und deine Kommilitonen gegeneinander in Teams antreten. Dabei werdet ihr interessante, schöne und verstörende Ecken Tübingens kennen lernen, dabei hoffentlich die Orientierung in eurer neuen Heimat etwas verbessern und Kontakte knüpfen.
	 Falls sich Details ändern sollten, findest du weitere Infos wie Uhrzeit und Treffpunkt auf unserer Webseite \url{https://www.fsi.uni-tuebingen.de/ersti/}.
\fi

%Wanderung #1
\ifml
    \item[Sunday, October 25th \YEAR, 11:00, in front of Neckarmüller]\ \\
        On a leisurely hike you will get to know not only your fellow students, but also a few lecturers and the worthwhile surroundings of Tübingen! The meeting point is at the Neckarbrücke \emph{in front of the} restaurant "`Neckarmüller"`.\\
        \seticon{faBus}~\textbf{bus stop:} Linie 1-22 ''Neckarbrücke''
\else
    \item[Sonntag, 25. Oktober \YEAR, 11 Uhr, vor dem Neckarmüller]\ \\
        Bei einer gemütlichen Wanderung lernt ihr neben euren Kommilitonen und Kommilitoninnen auch noch ein paar Dozierende und die sehenswerte Tübinger Umgebung kennen! Der Treffpunkt ist bei der Neckarbrücke (\emph{vor} dem Gasthaus „Neckarmüller“).\\
	    \seticon{faBus}~\textbf{Bushaltestelle:} Linie 1-22 ''Neckarbrücke''
\fi

% Kneipentour
\ifml
	\item[Friday, October 30th \YEAR, \textasciitilde 19:00, \footnotesize{location \& start time will be given to you after registration}]\ \\
        Tübingen is laced with small bars and pubs that have a significant impact on the night life in Tübingen. In order to calm down from the stress of this information-filled day, we'd like to invite you to go bar-hopping with us. We'll divide up into small groups and visit the different bars in the historic town center. Please bring enough cash, most places we will visit don't accept cards (\emph{none whatsoever}).

\else
	\item[Samstag, 30. Oktober \YEAR, \textasciitilde 19 Uhr, \footnotesize{Ort \& Startzeit wird dir nach Anmeldung mitgeteilt}]\ \\
        Tübingen ist übersät mit kleinen Kneipen und Bars, die das Nachtleben maßgeblich beeinflussen. Um den Stress des informationsgefüllten Tages etwas sacken zu lassen, laden wir dich zu einer ausgiebigen Kneipentour ein, bei der wir in Kleingruppen die verschiedenen Lokalitäten der Tübinger Altstadt besuchen. Bitte bringe genügend Bargeld mit, man kann in fast keiner der Tübinger Bars mit EC-Karte zahlen! -- Volksbanken und Sparkassen finden sich bei Bedarf in der Stadt.
\fi

%Online Stadtrallye
\ifml
    \item[Sonntag, November 1st \YEAR, 15:00, online]\ \\
        This semester the first online city rally is planned. Come by and get to know the city and the university virtually. You will receive the link in time after registration.
\else
    \item[Sonntag, 1. November \YEAR, 15 Uhr, online]\ \\
        Dieses Semester ist die erste Online Stadtrallye geplant. Komm vorbei und lerne mit deinen die Stadt und die Uni virtuell kennen. Den Link dazu bekommst du nach einer Anmeldung rechtzeitig zugeschickt.
\fi

\ifbachelor
    \iflehramt
        \item[Dienstag, 3. November \YEAR, 14 Uhr, online]\ \\
            Deine erste Vorlesung beginnt. \\
            Du hast um 14 Uhr „Informatik I“ bei \Infoprof.
            Alles, was du heute (und in Zukunft) benötigst: persönlichen Wachmacher, einen Stift und einen Block. %und den Studierendenausweis.

            %\seticon{faBus}~\textbf{Bushaltestelle:} BG Unfallklinik (Linie 5, 13, 14, 17, 18, 19, X15)
    \else
        \item[Montag, 2. November \YEAR, 8 Uhr, online]\ \\
            Deine erste Vorlesung beginnt. \\
            Du hast um 8 Uhr „Mathematik I“ bei \Matheprof.
            Alles, was du heute (und in Zukunft) benötigst: persönlichen Wachmacher, einen Stift und einen Block. %und den Studierendenausweis.

            %\seticon{faBus}~\textbf{Bushaltestelle:} BG Unfallklinik (Linie 5, 13, 14, 17, 18, 19, X15)
    \fi
\fi

%Spieleabend 1
%\ifml
%	\item[Friday, October 4th, \YEAR, 19:00, Sand]\ \\
%	We'd like to invite you to (an analog) board game evening with relaxed atmosphere at the Sand. We'll provide some games as well as drinks and snacks (for a small donation). Even though our collection is growing steadily, we're more than happy if you bring along your own games! Further details will be provided at \url{https://www.fsi.uni-tuebingen.de/ersti}.

%	\seticon{faBus}~\textbf{bus stop:} route 2, route 6 "`Sand Drosselweg"' (signposted from there)
%\else
%    \item[Donnerstag, 23. April \YEAR, 18:00 Uhr, Sand 1 A301]\ \\
%	Wir möchten dich zu einem kleinen (analog-) Spieleabend mit guter Gesellschaft und entspannter Atmosphäre auf dem Sand einladen. Für einige Spiele sowie Getränke und Knabberkram (gegen einen kleinen Obolus) sorgt die Fachschaft. Wir freuen uns natürlich sehr, wenn du auch eigene Spiele mitbringst, obwohl unsere Sammlung schon beachtlich ist! Details werden noch auf \url{https://www.fsi.uni-tuebingen.de/ersti/} bekannt gegeben.

%	\seticon{faBus}~\textbf{Bushaltestelle:} Linie 2, Linie 6, Sand Drosselweg (Rest ausgeschildert)
%\fi

%Filmabend
%\ifml
%	\item[Tuesday, October 2nd, \YEAR, 19:00, Sand 14, room A104 (meeting point is signposted)]\ \\
%	On this evening still early into the semester, we would like to invite you to a cozy movie night at the Sand. This is a perfect opportunity to relax, get to know the Sand, meet the people of the student council and future co-eds while watching a movie \footnote{Due to license restrictions, we're not allowed to tell you what movie wil be shown}.

%	\seticon{faBus}~\textbf{bus stop:} route 2, route 6 "`Sand Drosselweg"' (signposted from there)
%\else
	%\item[Dienstag, 31. März \YEAR, 19:00 Uhr, Sand 1 A301]\ \\
    %Am Abend des zweiten Vorkurstages möchten wir dich zu einem gemütlichen Filmabend auf dem Sand einladen.
	%Hier hast du die Möglichkeit, bei einem Film \footnote{welcher Film gezeigt wird, dürfen wir aus lizenzrechtlichen Gründen nicht bekannt geben.} zu entspannen, einige Fachschaftler, den Sand und eure zukünftigen Kommilitonen kennen zu lernen.

	%\seticon{faBus}~\textbf{Bushaltestelle:} Linie 2, Linie 6, Sand Drosselweg (Rest ausgeschildert)
%\fi

% Bus-Schnitzeljagd, neu im WS19/20
%\ifml
%	\item[Monday, October 7th \YEAR, 19:00, \textbf{in front of} Neckarmüller]\ \\
%    During a scavenger hunt you will be sent off in teams to explore the Tübingen bus network. By solving various puzzles, you will not only get to know your new fellow students better, but also get to know some stops that you will encounter more or less frequently in your everyday university life. Since students in Tübingen (and in the complete naldo area) are allowed to take the bus and train free of charge from Monday to Friday from 19:00 \footnote{as well as all day on Saturdays, Sundays and public holidays in Baden-Württemberg} onwards, you only need your student ID card for your exploration tour.\\
%If details change, you will find further information such as time and meeting point on our website \url{https://www.fsi.uni-tuebingen.de/ersti/}.\\

%\else
%	\item[Montag, 7. Oktober \YEAR, 19:00 Uhr, vor dem Neckarmüller]\ \\
%    Bei der Schnitzeljagd wirst du mit deinen Kommilitonen in Teams losgeschickt, um das Tübinger Busnetz zu erkunden. Durch das Lösen verschiedener Rätsel lernt ihr dabei nicht nur eure neuen Komilitonen sondern auch einige Haltestellen besser kennen, die euch in eurem Uni-Alltag mehr oder weniger häufig begegnen werden. Da Studenten in Tübingen (und im kompletten naldo-Bereich) montags bis freitags ab 19:00 Uhr \footnote{sowie ganztägig an Samstagen, Sonntagen und gesetzlichen Feiertagen in Baden-Württemberg} kostenlos Bus und Bahn fahren dürfen, benötigt ihr für eure Erkundungstour lediglich euren Studierendenausweis.\\
%    Falls sich Details ändern sollten, findest du weitere Infos wie Uhrzeit und Treffpunkt auf unserer Webseite \url{https://www.fsi.uni-tuebingen.de/ersti}.\\
%\fi

%Grillen
%\ifml
%	\item[Tuesday, October 15th \YEAR, 17:00, Sand 13, garden]\ \\
%	You're not in the mood for cooking this evening? Fear not! The student coucil invites you for a BBQ. Bring whatever you want to put on the grill, please also bring your own plates and cutlery. The garden has enough space for stuff like volleyball, football etc. as well.
%	If the schedule should change somehow, you can find updated details on our website at \url{https://www.fsi.uni-tuebingen.de/ersti/}.

%	\seticon{faBus}~\textbf{bus stop:} route 2, route 6, "`Sand Drosselweg"' (signposted from there)
%\else
	%\item[Donnerstag, 2. April \YEAR, 18:00 Uhr, im Garten des Sandes ]\ \\
	%Du hast keinen Bock auf Kochen? Dann bist du hier genau richtig! In geselliger Runde wird die Fachschaft mit dir grillen. Bringt dazu mit, was auch immer du zum Grillen brauchst, Gas- und Kohlegrill warten auf dich. Bring bitte auch dein Besteck und Geschirr selbst mit!\\
	 %Auf dem Sand ist es auch möglich, Volleyball, Fußball, usw. zu spielen. Wir freuen uns auf dich!
	%Falls sich Details ändern sollten, findest du weitere Infos wie Uhrzeit und Treffpunkt auf unserer Webseite \url{https://www.fsi.uni-tuebingen.de/ersti/}.

	%\seticon{faBus}~\textbf{Bushaltestelle:} Linie 2, Linie 6, Sand Drosselweg (Rest ausgeschildert)
%\fi



% Frühstück
%\ifml
% Frühstück Master ML
%    \item[Friday, October 11th \YEAR, 9:00, Mensa Morgenstelle]\ \\
%        Let's have breakfast together, we're paying! You will get to know a lot of stuff about the university, the student council, what it's like to study in Tübingen and what you can expect in the upcoming months -- especially when talking to older co-eds.
%    \ifwintersemester You will also be greeted by \Infoprof~-- he will lecture Informatics I. This is a lecture for bachelor students, but maybe you will get to know him in other lectures as well. \fi
%    Afterwards, we will provide a tour of the Morgenstelle so you get to know the most important rooms and lecture halls.
%    You can reach the Morgenstelle via bus, bicycle or on foot. However, the way up to the Morgenstelle is quite steep -- Tübingen is hilly (German: \emph{hügelig}), so you should factor in the time you need to get there.
%    The simplest (but also steepest) way is via "`Parkhaus König"' and up "`Schnarrenbergstraße"', past the university hospital. If you're starting at Waldhäuser Ost, follow the "`Nordring"' in direction of "`Uni-Morgenstelle"' / "`Kliniken Berg"'.
%    If you want to drive there by car, you can park (not free!) by the side of the road on Nordring or inside the multi-story cark park "`Ebenhalde"' north of the Morgenstelle. However, the simplest way is via bus.
%    From 11:45 onwards, you can try out the mensa food if you like.\\
%    \seticon{faBus}~\textbf{bus stop:} BG Unfallklinik (bus routes 5, 13, 14, 17, 18, 19, X15)

%\else
    %\item[Donnerstag, 9. April \YEAR, 9:00 Uhr, Sand 1 A104]\ \\
        %Wir laden dich an diesem Morgen zu einem gemütlichen Frühstück ein! Dabei erfährst du einiges über die Uni, die Fachschaft und was dich in den nächsten Monaten erwartet -- auch im Gespräch mit älteren Studierenden.
%    \ifwintersemester Außerdem wirst du durch \Infoprof~-- er wird die Informatik I Vorlesung halten -- begrüßt. \fi
    %\ifsommersemester Außerdem wirst du durch \Infoprof~-- er wird die Informatik II Vorlesung halten -- begrüßt. \fi
    %\ifmaster Zwar ist Informatik I eine Bachelor-Veranstaltung, aber du wirst \Infoprof~ vielleicht auch in Master-Vorlesungen kennen lernen. \fi
%    \ifwintersemester
%        Danach machen wir eine Führung über die Morgenstelle, damit du die wichtigsten Räume und Hörsäle kennen lernst.
%        Zur Morgenstelle kommt man entweder mit dem Bus, zu Fuß oder mit dem Rad. Da der Weg zur Morgenstelle aber sehr steil ist (Tübingen ist hügelig), sollte man hierfür einiges an Zeit einrechnen.
%        Der einfachste Weg ist hier über das Parkhaus König. Von dort musst du bergauf der Schnarrenbergstraße folgen. Es geht dann zunächst an den Uni-Kliniken Berg vorbei, anschließend erreichst du die Morgenstelle. Falls du aus der Richtung Waldh\"auser-Ost kommt, so musst du dem Nordring in Richtung Kliniken Berg folgen. Für beide Wege solltest du jeweils mindestens eine halbe Stunde zu Fuß einrechnen.
%        Wenn du mit dem Auto ankommst, kannst du (kostenpflichtig) an den Straßenseiten des Nordrings, oder im Parkhaus "`Ebenhalde"' oberhalb der Morgenstelle parken. Am einfachsten geht es jedoch mit dem Bus.
%        Wenn du Lust hast, kannst du ab 11:45 Uhr das Mensaessen ausprobieren.
%    \fi

%    \ifwintersemester \seticon{faBus}~\textbf{Bushaltestelle:} BG Unfallklinik (Linie 5, 13, 14, 17, 18, 19, X15) \fi
    %\ifsommersemester \seticon{faBus}~\textbf{Bushaltestelle:} Linie 2, Linie 6, Sand Drosselweg \fi
%\fi

\iflehramt
    \ifbachelor
        %\item[Donnerstag, 9. April \YEAR, 10:00-12:00 Uhr, Kupferbau, Hörsaal 22]\ \\
            %Anfis im \textbf{Bachelor of Education} können beim Frühstück (rein theoretisch) nicht mit dabei sein, denn parallel zum Frühstück werdert ihr um 10:00 Uhr über die speziellen Anforderungen des Lehramtsstudiums informiert. Neben den eigentlichen Fachinhalten kommen im Bachelor of Education noch einige andere Dinge auf euch zu, z.B. ein Orientierungspraktikum, die Fachdidaktik und der Studienbereich Bildungswissenschaften. Zu Beginn des Masters steht dann ein Schulpraxissemester an, und nach dem Studienabschluss das Referendariat. Bei dieser Veranstaltung werden euch alle diese Elemente des Lehramtsstudiums vorgestellt.

        %\seticon{faBus}~\textbf{Bushaltestelle:} Hölderlinstraße bzw. Uni/Neue Aula
    \fi
    \ifmaster
        %\item[Donnerstag, 9. April \YEAR, 14:00-16:00 Uhr, Kupferbau, Hörsaal 22]\ \\
            %Wenn du im \textbf{Master of Education} studierst, kannst du zum Frühstück bleiben, eure Einführungsveranstaltung beginnt dann erst um 14 Uhr. Hier bekommst du u.a. einen Einblick in die verschiedenen Studienanteile des Master of Education und Informationen zum Schulpraxissemester.
    \fi
\fi
%Begrüßung Online
%\item[Freitag, 30. Oktober \YEAR, 10:00 Uhr, via Zoom]
%An diesem Tag werdet ihr von den Professoren aus unserem Fachbereich begrüßt. Nähere Details folgen.


%\ifwintersemester
%    \ifml
%	    \item[Friday, October 11th, \YEAR, 14:00, Sand, Foyer (rooms and schedule will follow)]\ \\
%    	At 2 o'clock, we'll meet up again at the Sand, home of the faculty of computer science. % Ich weiß dass es keine Fakultät ist, übersetz mir mal einer Fachbereich...
%    	Here you can gain some interesting insights into the work of the different work groups present at the Sand.
%	    Getting to know the work groups makes sense for ML as well, depending on how you want to focus your studies or thesis. Depending on your interests you can participate in different mini workshops or listen to talks, however, this requires some planning on your part. The talks available (as well as the time slots) are subject to change, please refer to \url{https://www.fsi.uni-tuebingen.de/ersti/} the day before.
%    	For the most part, the work groups focusing on Machine Learning are located at the Tübinen AI research building, however, access there is rather restricted.
%
%	    \seticon{faBus}~\textbf{bus stop:} route 2, route 6, "`Sand Drosselweg"' (signposted from there)
%    \else
%	    \item[Freitag, 11. Oktober \YEAR, 14 Uhr, Sand 1, Foyer (Räume und Programm folgen)]\ \\
%    	Um 14:00 treffen wir uns auf dem Sand (dem Sitz der Informatik in Tübingen) wieder. Hier werden wir dich mit Informationen rund um das Studium und die Arbeit unserer Lehrstühle auf dem Sand versorgen.
%    	Nachdem du über den Verlauf des Studiums der ersten Wochen, Monate und Semester informiert wurdest, bieten dir die verschiedenen Fachbereiche mit interaktiven Vorträgen einen Einblick in ihre Arbeit. Die Fachbereiche kennen zu lernen lohnt sich für Studierende aller Studiengänge. Die Vorträge zeigen, welche Themen am Wilhelm-Schickard-Institut verfolgt werden, geben dir ein Gefühl, was für dich hier interessant sein kann und wo du möglicherweise Schwerpunkte im Studium, einer Studien- oder Abschlussarbeit setzen willst. Je nach Interesse kannst du dir verschiedene Vorträge anhören, allerdings bedarf dies an Auswahl und Planung von deiner Seite. Die angebotenen Vorträge können sich thematisch oder, wie die Begrüßungen, im Zeitrahmen ändern. Schau daher auch kurzfristig (am selben Tag) auf \url{https://www.fsi.uni-tuebingen.de/ersti/} nach.
%
%        \seticon{faBus}~\textbf{Bushaltestelle:} Linie 2, Linie 6, Sand Drosselweg (Rest ausgeschildert)
%    \fi
%\fi

%Mastercafé
%\ifmaster
%    \ifml
%        \item[Friday, October 11th \YEAR, 11:00, Maria-von-Linden-Straße 6]\ \\
%    At this informal meeting you will have the opportunity to get to know your fellow students and some of the lecturers over a coffee. And then there will be cake.\footnote{This is not a lie!}

%        \seticon{faBus}~\textbf{bus stop:} route 3, "`Maria-von-Linden-Straße"' (signposted from there)
%    \else
    %\item[Dienstag, 14. März \YEAR, 16:00 Uhr, Sand 1 A301]\ \\
    %Bei diesem informellen Treffen hast du Gelegenheit, neue Master-Studierende aus deinem eigenen und auch aus anderen Studiengängen sowie einige der Dozenten bei Kaffee kennenzulernen. Und dann gibt es Kuchen. \footnote{Das ist keine Lüge!}

        %\seticon{faBus}~\textbf{Bushaltestelle:} Linie 2, Linie 6, Sand Drosselweg (Rest ausgeschildert)
%    \fi
%\fi

%Um 13{\ifkogwiss}:45{\fi} Uhr treffen wir uns auf dem „Sand“ (dem Sitz des Wilhelm-Schickard-Instituts)
%wieder. Hier werden wir euch mit Informationen rund um{\ifkogwiss} {\else} euer Studium und {\fi}die Arbeit unserer Lehrstühle auf dem Sand versorgen.
%
%{\ifkogwiss}Als Studierende der Kognitionswissenschaften könnt ihr um 13:45 Uhr zusto{\ss}en, die anderen Studiengänge erhalten vorab eine spezifische Einführung, die für euch Montag, 15. Oktober am Psychologischen Institut stattfindet. {\else}{\ifinfo}Als Lehramt Informatik und Informatik-Studierende werden euch auch mögliche Nebenfächer vorgestellt, ihr werdet {\else}Ihr werdet als *-Informatik-Studierende {\fi}noch einmal spezifisch begrüßt und ihr erhaltet einen studiengangsspezifischen Einblick in Forschung und Lehre.{\fi}
%
%{\ifkogwiss}Die {\else}Nachdem ihr über den Verlauf des Studiums der ersten Wochen, Monate und Semester informiert wurdet, bieten euch die {\fi}verschiedenen Fachbereiche {\ifkogwiss}auf dem Sand bieten euch {\else} {\fi}mit Vorträgen einen Einblick in ihre Arbeit. Die Fachbereiche kennen zu lernen lohnt sich für Studierende aller Studiengänge, die Vorträge zeigen welche Themen am Wilhelm-Schickard-Institut verfolgt werden, geben euch ein Gefühl, was für euch hier interessant sein kann und wo ihr möglicherweise Schwerpunkte im Studium, einer Studien- oder Abschlussarbeit setzten wollt. Je nach Interesse könnt ihr euch verschiedene Vorträge anhören, allerdings bedarf dies an Auswahl und Planung von eurer Seite. Die angebotenen Vorträge können sich thematisch oder, wie die Begrüßungen, im Zeitrahmen ändern. Schaut daher auch kurzfristig (am selben Tag) auf \url{https://www.fsi.uni-tuebingen.de/ersti/} nach.
%
%\seticon{faBus}~\textbf{Bushaltestelle:} Sand Drosselweg (Rest ausgeschildert)


%\item[Freitag, 12. Oktober \YEAR, 13{\ifkogwiss}:45{\fi} Uhr, Sand (Räume und Programm folgen)]\ \\
%Um 13{\ifkogwiss}:45{\fi} Uhr treffen wir uns auf dem „Sand“ (dem Sitz des Wilhelm-Schickard-Instituts)
%wieder. Hier werden wir euch mit Informationen rund um{\ifkogwiss} {\else} euer Studium und {\fi}die Arbeit unserer Lehrstühle auf dem Sand versorgen.
%
%{\ifkogwiss}Als Studierende der Kognitionswissenschaften könnt ihr um 13:45 Uhr zusto{\ss}en, die anderen Studiengänge erhalten vorab eine spezifische Einführung, die für euch Montag, 15. Oktober am Psychologischen Institut stattfindet. {\else}{\ifinfo}Als Lehramt Informatik und Informatik-Studierende werden euch auch mögliche Nebenfächer vorgestellt, ihr werdet {\else}Ihr werdet als *-Informatik-Studierende {\fi}noch einmal spezifisch begrüßt und ihr erhaltet einen studiengangsspezifischen Einblick in Forschung und Lehre.{\fi}
%
%{\ifkogwiss}Die {\else}Nachdem ihr über den Verlauf des Studiums der ersten Wochen, Monate und Semester informiert wurdet, bieten euch die {\fi}verschiedenen Fachbereiche {\ifkogwiss}auf dem Sand bieten euch {\else} {\fi}mit Vorträgen einen Einblick in ihre Arbeit. Die Fachbereiche kennen zu lernen lohnt sich für Studierende aller Studiengänge, die Vorträge zeigen welche Themen am Wilhelm-Schickard-Institut verfolgt werden, geben euch ein Gefühl, was für euch hier interessant sein kann und wo ihr möglicherweise Schwerpunkte im Studium, einer Studien- oder Abschlussarbeit setzten wollt. Je nach Interesse könnt ihr euch verschiedene Vorträge anhören, allerdings bedarf dies an Auswahl und Planung von eurer Seite. Die angebotenen Vorträge können sich thematisch oder, wie die Begrüßungen, im Zeitrahmen ändern. Schaut daher auch kurzfristig (am selben Tag) auf \url{https://www.fsi.uni-tuebingen.de/ersti/} nach.
%
%\seticon{faBus}~\textbf{Bushaltestelle:} Sand Drosselweg (Rest ausgeschildert)

% erste Vorlesung
%\ifbachelor
%\item[Dienstag, 14. April \YEAR, Morgenstelle, Hörsaal N7]\ \\
%Deine erste Vorlesung beginnt um
%\ifwintersemester 8 Uhr -- Du hast „Mathe I für Informatiker“  \fi
%\ifsommersemester 10 Uhr -- Du hast „Mathe II für Informatiker“  \fi
%bei \Matheprof.
%Alles, was du heute (und in Zukunft) benötigst: persönlichen Wachmacher, einen Stift, einen Block und den Studierendenausweis.

%\seticon{faBus}~\textbf{Bushaltestelle:} BG Unfallklinik (Linie 5, 13, 14, 17, 18, 19, X15)
%\fi

%Akadem. Spieleabend
%\ifml
%    \item[Thursday, October 17th, \YEAR 18:00 Sand 1, A301]\ \\
%        On this afternoon/evening we'd first like to invite you a second time to play board games with your fellow students. As soon as the time has advanced far enough, we'd like to live it up in the pubs.\\
%	\seticon{faBus}~\textbf{bus stop:} route 2, route 6 "`Sand Drosselweg"' (signposted from there)
%\else
    %\item[Donnerstag, 16. April \YEAR, 19:00 Uhr, Sand 1 A301]\ \\
%An diesem Nachmittag/Abend möchten wir dich zunächst auf den Sand einladen, um in gemütlicher Runde mit anderen Kommilitonen und Fachschaftlern Brett- und Gesellschaftsspiele zu spielen. Sobald die Zeit an diesem Abend ausreichend fortgeschritten ist, möchten wir zusammen mit dir ein wenig die Kneipen der Altstadt unsicher machen.

%\seticon{faBus}~\textbf{Bushaltestelle:} Linie 2, Linie 6, Sand Drosselweg (Rest ausgeschildert)
%\fi

%Online Spieleabend
\ifml
    \item[Freitag, November 6th \YEAR, 19:00, online]\ \\
    Today the online game evening of the student council, which was introduced especially for Corona conditions, takes place. Here you meet your fellow students on Discord and play Fever.io, Drawful2.io and more! A registration is not necessary. You will find the link to the Discord server on our website in time.
\else
    \item[Freitag, 6. November \YEAR, 19 Uhr, online]\ \\
        Heute findet der speziell für Corona-Bedingungen eingeführte Online-Spieleabend der Fachschaft statt. Hier kommst du auf Discord mit deinen Kommilitonen zusammen und spielst Fever.io, Drawful2.io und mehr! Eine Anmeldung ist nicht notwendig. Den Link zum Discord Server findest du rechtzeitig auf unserer Website.
\fi

%Wanderung #2
\ifml
    \item[Sonntag, November 8th \YEAR, 11:00, WHO (Waldhäuserstraße 122, 72076 Tübingen)]\ \\
    On a leisurely hike you will get to know not only your fellow students, but also a few lecturers and the worthwhile surroundings of Tübingen!\\
    \seticon{faBus}~\textbf{bus stop:} Linie 2, 3, 4, 5 ''Ulmenweg'', WHO
\else
    \item[Sonntag, 08. November \YEAR, 11 Uhr, WHO (Waldhäuserstraße 122, 72076 Tübingen)]\ \\
        Bei einer gemütlichen zweiten Wanderung lernt ihr neben euren Kommilitonen und Kommilitoninnen auch noch ein paar Dozierende und die sehenswerte Tübinger Umgebung kennen!\\
        \seticon{faBus}~\textbf{Bushaltestelle:} Linie 2, 3, 4, 5 ''Ulmenweg'', WHO
\fi

%\ifkogwiss
%\item[Montag, 14. Oktober \YEAR, Uhrzeit und Ort TBA]\ \\
%    Hier stellen sich die kognitionswissenschaftlichen Lehrstühle (also die verschiedenen Forschungsbereiche der Professoren) vor. Dies ist super um einen Überblick zu bekommen, was die Kognitionswissenschaft alles beinhaltet und um erste Eindrücke von den Profs zu bekommen. Auch die Fachschaft stellt sich dir hier erstmals vor. Danach gibt es noch Gelegenheit, mit der Fachschaft ein Glas Milch trinken zu gehen. Auch für Kogni-Master ist das eine super Veranstaltung.
%\fi

%\ifkogwiss
 %   \item[Mittwoch, 16. Oktober, \YEAR, Uhrzeit und Ort TBA]\ \\
  %       Damit man sich auch unter den Kognis kennenlernen kann, veranstalten wir einen Spiele- und Informationsabend für die Kogni-Anfis. Hier können Gesellschaftsspiele und bei gutem Wetter auch Tischtennis und Volleyball gespielt werden. Für Verpflegung können wir leider nicht auf eigene Kosten sorgen aber wir stellen gegen eine kleine Spende Getränke bereit und bestellen Pizza. Es werden auch einige höhersemestrige Kognis und Fachschaftler da sein, die man zum Kogni-Studium ausfragen kann. Kogni-Master sind natürlich auch herzlich eingeladen.
%	\seticon{faBus}~\textbf{Bushaltestelle:} Linie 2, Sand Drosselweg (Rest ausgeschildert)
%\fi

\ifkogwiss
% Bus-Schnitzeljagd, neu im WS19/20
\item[Samstag, 31. Oktober \YEAR, mit Anmeldesystem]\ \\
    Bei der Schnitzeljagd wirst du mit deinen Kommilitonen in Teams losgeschickt, um das Tübinger Busnetz zu erkunden. Durch das Lösen verschiedener Rätsel lernt ihr dabei nicht nur eure neuen Komilitonen sondern auch einige Haltestellen besser kennen, die euch in eurem Uni-Alltag mehr oder weniger häufig begegnen werden. Da Studenten in Tübingen (und im kompletten naldo-Bereich) montags bis freitags ab 19:00 Uhr sowie ganztägig an Samstagen, Sonntagen und gesetzlichen Feiertagen in Baden-Württemberg kostenlos Bus und Bahn fahren dürfen, benötigt ihr für eure Erkundungstour lediglich euren Studierendenausweis.\\
    Für die Durchführung planen wir ein Anmeldesystem, um die Gruppen gestaffelt losschicken zu können. Dieses findest du über einen Link auf unserer Fachschafts-Website \url{https://www.fs-kogni.uni-tuebingen.de/}.
    Falls sich Details ändern sollten, findest du weitere Infos wie Uhrzeit und Treffpunkt ebenfalls auf unserer Webseite oder auf der Website der Fachschaft Informatik \url{https://www.fsi.uni-tuebingen.de/ersti}.\\
\fi

\ifkogwiss
\item[Montag, 02. November \YEAR, 16:00 Uhr und online]\ \\
    Hier stellen sich die kognitionswissenschaftlichen Lehrstühle (also die verschiedenen Forschungsbereiche der Professoren) vor. Dies ist super um einen Überblick zu bekommen, was die Kognitionswissenschaft alles beinhaltet und um erste Eindrücke von den Profs zu bekommen. Auch die Fachschaft stellt sich dir hier erstmals vor. %Danach gibt es noch Gelegenheit, mit der Fachschaft ein Glas Milch trinken zu gehen.
    Auch für Kogni-Master ist das eine super Veranstaltung. \\
    Falls sich auf Grund der aktuellen Situation Details ändern sollten, findest du weitere Infos auf der Website \url{https://www.fs-kogni.uni-tuebingen.de/}.
    
\fi

\ifkogwiss
    \item[Mittwoch, 04. November, \YEAR, 20:00 Uhr und Ort Sand 14]\ \\
         Damit man sich auch unter den Kognis kennenlernen kann, veranstalten wir einen Spiele- und Informationsabend für die Kogni-Anfis. Dabei werden auch einige höhersemestrige Kognis und Fachschaftler da sein, die man zum Kogni-Studium ausfragen kann. Bei gutem Wetter werden wir möglichst draußen sitzen und uns so der aktuellen Situation anpassen. Wir hoffen, dass wir zusammen einen schönen Abend verbringen können. Kogni-Master sind natürlich auch herzlich eingeladen.
         %Hier können Gesellschaftsspiele und bei gutem Wetter auch Tischtennis und Volleyball gespielt werden. Für Verpflegung können wir leider nicht auf eigene Kosten sorgen aber wir stellen gegen eine kleine Spende Getränke bereit und bestellen Pizza. Es werden auch einige höhersemestrige Kognis und Fachschaftler da sein, die man zum Kogni-Studium ausfragen kann. Kogni-Master sind natürlich auch herzlich eingeladen.
	\seticon{faBus}~\textbf{Bushaltestelle:} Linie 2, Sand Drosselweg (Rest ausgeschildert)
	Falls sich auf Grund der aktuellen Situation Details ändern sollten, findest du weitere Infos auf der Website \url{https://www.fs-kogni.uni-tuebingen.de/}.
\fi

%Clubhausfest
\ifml
    \item[Thursday, January 14th 2021, 21:00, Clubhaus]\ \\
        Every Thursday during the lecture period, a different student body or group of students organizes the "`Clubhaus Fest"'\footnote{a university building where a different student council hosts a party every thursday} in the Clubhaus, directly opposite the New auditorium. Today the attendance is particularly worthwhile, because the student councils of computer science, cognitive science and psychology are hosts! We are looking forward to meeting you!(This event is scheduled to take place, but the circumstances may still change because of the corona virus!)

        \seticon{faBus}~\textbf{bus stop:} Uni/Neue Aula or Hölderlinstraße
\else
    \item[Donnerstag, 14. Januar 2021, 21:00 Uhr, Clubhaus]\ \\
        Während der Vorlesungszeit richtet jeden Donnerstag eine andere Fachschaft oder studentische Gruppierung das Clubhausfest im Clubhaus, direkt gegenüber der Neuen Aula, aus. Heute lohnt sich der Besuch jedoch ganz besonders, denn die Fachschaften Informatik, Kogni und Psychologie sind Gastgeber! Wir freuen uns auf dich! (Dieses Ereignis findet vorraussichtlich statt, jedoch kann sich die Situation wegen Corona immer noch ändern!)
%Das anfängliche Chaos des Vorlesungsbeginns hat sich gelegt und auch an diesem Donnerstag findet das Clubhausfest statt.

%\seticon{faBus}~\textbf{Bushaltestelle:} Uni/Neue Aula bzw. Hölderlinstraße
\fi

%Anfiwochenende
%\ifml
%    \item[Friday, October 18th, \YEAR, 13:00, train station]\ \\
%        If you are still motivated to meet new people after a 2 week Anfi program and the first week of university, you should spend a weekend with us in Dettingen. There you will find the "`Naturfreundehaus"' which we rented from October 18th to October 20th. On this weekend we would like to prepare you a bit more for your everyday studies and help you to get to know each other. The open slots for this weekend are limited and will be allocated quickly. If you want to participate, just send a mail to Roman (\texttt{roman.schulte\At student.uni-tuebingen.de}). For the payment of board and lodging we expect a contribution of EUR 20,- per participant. Further information will soon be available on our homepage.

%\else
%    \item[Freitag, 18. Oktober \YEAR, 13:00 Uhr, Bahnhof]\ \\
%        Falls du nach 2 Wochen Anfi-Programm und der ersten Woche Uni immer noch motiviert bist, neue Leute kennen zu lernen, dann verbringe doch mit uns ein Wochenende in Dettingen. Dort befindet sich das Naturfreundehaus, das wir vom 18.10. bis zum 20.10. gemietet haben. An diesem Wochenende möchten wir dich noch ein wenig mehr auf den Studienalltag vorbereiten und euch helfen, untereinander Kontakte zu knüpfen. Die Plätze für dieses Wochenende sind begrenzt und werden schnell vergeben. Wenn du teilnehmen möchtest, sende einfach eine Mail an Roman (\texttt{roman.schulte\At student.uni-tuebingen.de}). Um die Unterkunft und Verpflegung zu bezahlen, rechnen wir mit einem Unkostenbeitrag von \EUR{20} pro Teilnehmer. Weitere Infos findest sich in Kürze auch auf unserer Homepage.
%\fi

%Konfigabend
%\ifwintersemester
%    \ifml
%        \item[Tuesday, October 22nd, \YEAR, 19:00, Sand 1, A301]\ \\
%            On this evening we'd like to offer you the opportunity to configure your laptop computer for everyday life at the university. Also there shall be no lack of opportunity to revel in memories\footnote{https://xkcd.com/422/}.\\
%            \textbf{please remember to bring the power cord to your laptop.}

%	    \seticon{faBus}~\textbf{bus stop:} route 2, "`Sand Drosselweg"' (signposted from there)
%    \else
%        \item[Dienstag, 22. Oktober \YEAR, 19:00 Uhr, Sand 1, A301]\ \\
%            An diesem Abend möchten wir dir die Gelegenheit bieten, eure Laptops für den Uni-Alltag einzurichten. Zudem soll die Gelegenheit nicht ausbleiben, in alten Erinnerungen zu schwelgen\footnote{https://xkcd.com/422/}.\\
%            \textbf{Denkt bitte daran, die Stromkabel eurer Laptops mitzubringen.}

%        \seticon{faBus}~\textbf{Bushaltestelle:} Linie 2, Sand Drosselweg (Rest ausgeschildert)
%    \fi
%\fi

%Kneipentour II
%\ifml
%    \item[Friday, October 25th, \YEAR, 19:00, Neckarbrücke]\ \\
%        Because Tübingen's pub landscape is manifold, it is not possible to visit all of them in one night. Therefore we'd like to invite you a second time to roam the pubs with us. We meet again on the Neckarbrücke in front of the Neckarmüller. Remember to bring enough cash, as cards will probably not be accepted!
%\else
%    \item[Freitag, 25. Oktober \YEAR, 19:00 Uhr, Neckarbrücke]\ \\
%        Weil die Kneipenlandschaft in Tübingen so mannigfaltig ist, dass es an einem Abend unmöglich ist, überall mal gewesen zu sein, gibt es an diesem Abend eine Fortsetzung. Treffpunkt ist wieder an der Neckarbrücke vor dem Neckarmüller. Auch hier gilt: bitte genug Bargeld mitnehmen!
%\fi

\ifkogwiss

    \item[Montag, 8. Oktober \YEAR, 17 Uhr, Sand Terasse ]\ \\
    Damit sich die Kognis untereinander kennenlernen, gibt es einen Abend nur für diese. Der genaue Ablauf des Abends wird im Laufe der ersten Vorkurswoche bekannt gegeben.
\fi

%\item[Freitag, 12. Oktober \YEAR, 13{\ifkogwiss}:45{\fi} Uhr, Sand (Räume und Programm folgen)]\ \\
%Um 13{\ifkogwiss}:45{\fi} Uhr treffen wir uns auf dem „Sand“ (dem Sitz des Wilhelm-Schickard-Instituts)
%wieder. Hier werden wir euch mit Informationen rund um{\ifkogwiss} {\else} euer Studium und {\fi}die Arbeit unserer Lehrstühle auf dem Sand versorgen.
%
%{\ifkogwiss}Als Studierende der Kognitionswissenschaften könnt ihr um 13:45 Uhr zusto{\ss}en, die anderen Studiengänge erhalten vorab eine spezifische Einführung, die für euch Montag, 15. Oktober am Psychologischen Institut stattfindet. {\else}{\ifinfo}Als Lehramt Informatik und Informatik-Studierende werden euch auch mögliche Nebenfächer vorgestellt, ihr werdet {\else}Ihr werdet als *-Informatik-Studierende {\fi}noch einmal spezifisch begrüßt und ihr erhaltet einen studiengangsspezifischen Einblick in Forschung und Lehre.{\fi}
%
%{\ifkogwiss}Die {\else}Nachdem ihr über den Verlauf des Studiums der ersten Wochen, Monate und Semester informiert wurdet, bieten euch die {\fi}verschiedenen Fachbereiche {\ifkogwiss}auf dem Sand bieten euch {\else} {\fi}mit Vorträgen einen Einblick in ihre Arbeit. Die Fachbereiche kennen zu lernen lohnt sich für Studierende aller Studiengänge, die Vorträge zeigen welche Themen am Wilhelm-Schickard-Institut verfolgt werden, geben euch ein Gefühl, was für euch hier interessant sein kann und wo ihr möglicherweise Schwerpunkte im Studium, einer Studien- oder Abschlussarbeit setzten wollt. Je nach Interesse könnt ihr euch verschiedene Vorträge anhören, allerdings bedarf dies an Auswahl und Planung von eurer Seite. Die angebotenen Vorträge können sich thematisch oder, wie die Begrüßungen, im Zeitrahmen ändern. Schaut daher auch kurzfristig (am selben Tag) auf \url{https://www.fsi.uni-tuebingen.de/ersti/} nach.
%
%\seticon{faBus}~\textbf{Bushaltestelle:} Sand Drosselweg (Rest ausgeschildert)


%\ifkogwiss
%\ifbachelor
%\item[Montag, 15. Oktober \YEAR, 17 Uhr, Psychologisches Institut, Hörsaal]\ \\
%An diesem Abend werdet ihr von Frau Prof. Rolke und Frau Jendreyko begrüßt. Zudem stellen sich euch, nach einer kurzen Einführung in den Studienaufbau, die Lehrstühle der Kognitionswissenschaft mit ihren Themen und Forschungsgebieten vor.
%\fi
%%\ifmaster
%%\item[Montag, 15. Oktober \YEAR, 15:00 Uhr, Psychologisches Institut, Seminarraum 4.326 ]\ \\
%%Heute Abend stellen sich euch die verschiedene Lehrstühle der Kognitionswissenschaft vor, um einen Einblick in ihre Forschungsthemen zu ermöglichen.
%%\fi
%
%\seticon{faBus}~\textbf{Bushaltestelle:} Hölderlinstraße / Uni-Kliniken Tal
%\fi
%
%\ifkogwiss
%\ifmaster
%\item[Montag, 15. Oktober \YEAR, 15:00 Uhr, Psychologisches Institut, Seminarraum 4.326 ]\ \\
%Hier bekommen speziell Master-Studenten eine separate Begrüßung durch Frau Prof. Rolke, die euch einen ersten Einblick ins Studium gibt, indem sie den regulären Studienaufbau des Master Kognitionswissenschaft erläutert. %Ort und Zeit werden noch auf \url{https://www.fsi.uni-tuebingen.de/ersti/} bekannt gegeben.
%\fi
%\fi

%Anfi-Mentorenprogramm
Dieses Semester bietet unser Fachbereich speziell ein Mentoren-Programm für Erstsemestler an. Dabei soll es ein regelmäßiges Treffen zwischen einem Professor und Kleingruppen an Studenten geben, in denen Fragen und Probleme geklärt werden. Weitere Details folgen.

\end{description}
