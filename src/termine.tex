\setlength{\fboxrule}{4pt}
	\fcolorbox{red}{white}{
		\begin{minipage}[t]{
			\textwidth}\textbf{Achtung!} Aufgrund der aktuellen Lage bezüglich COVID-19 können sich die Anfi-Termine für dieses Semester noch ändern. Wir sind im 				ständigen Austausch mit dem Fachbereich und der Universitätsleitung und müssen den weiteren Verlauf von Verboten und Richtlinien abwarten. 
			Schaut \textbf{auf jeden Fall} immer wieder auf \url{https://www.fsi.uni-tuebingen.de/ersti} nach, dort werden wir die aktuellsten Daten veröffentlichen, 			  sobald wir mehr wissen.
		\end{minipage}}
\begin{description}
  % HINWEIS ZUR ANMELDUNG - GILT IMMER - NICHT LÖSCHEN
  \ifml
    \item[Note:] Unless stated otherwise (i.e. in the description of the event itself), explicitly signing up for the events of the student council is usually \textbf{not necessary}. Details on the respective event can always be found at \url{https://www.fsi.uni-tuebingen.de/ersti} as well.
  \else
    \item[Hinweis:] Sofern nicht anders angegeben (z.B. im Text der jeweiligen Veranstaltung), ist eine explizite Anmeldung zu den Anfi-Veranstaltungen der Fachschaft normalerweise \textbf{nicht nötig}. Weitere Details zur jeweiligen Veranstaltung finden sich auch immer auf \url{https://www.fsi.uni-tuebingen.de/ersti}.
  \fi
  % ENDE HINWEIS ZUR ANMELDUNG

\ifkogwiss
    \ifmaster
        \item[Montag, 19. Oktober \YEAR, 08:00 Uhr, Ort Psychologisches Institut 4.332 und 4.326]\ \\
        Heute beginnt der Vorbereitungskurs Mathematik speziell für Kognitionswissenschaftler im Master. Es ist nicht Pflicht daran teilzunehmen, es ist aber sehr empfehlenswert. Nicht zuletzt lernt ihr hier erste Freunde kennen! Der Vorkurs bietet euch eine Zusammenfassung des Mathestoffs, der im Bachelor Kognitionswissenschaft an der Uni Tübingen behandelt wird.
        \textbf{Wenn ihr euren Bachelor nicht an der Universität Tübingen oder in einem anderen Fach erworben habt, kann der Vorkurs für euch sinnvoll sein.} Falls ihr Quereinsteiger seid, werdet ihr dadurch an die in eurem Studium benötigten mathematischen Grundlagen herangeführt. Falls ihr bereits mathematisches Vorwissen mitbringt, ist der Kurs eine gute Gelegenheit, euer Wissen aufzufrischen.\\
         Es wäre super, wenn ihr euch mit einer kurzen Mail an \texttt{p.fischer\At student.uni-tuebingen.de} bei Paul Fischer anmelden würdet. Als Treffpunkt für Montag den 19. Oktober könnt ihr euch am Eingang des Psychologischen Instituts einfinden. Alle weiteren Infos erhaltet ihr dann per Mail.\\

%Stattfinden wird der Vorkurs im ÜR 08 in der alten Physik, das ist die Gmelinstraße 6. Diese befindet sich an der Bushaltestelle Gmelinstraße, nördlich gegenüber von der Neuen Aula. Wenn ihr auf der Wilhelmstraße seid, geht rechts an der Neuen Aula vorbei, dort findet ihr die Alte Physik an der Ecke Gmelin- und Nauklerstraße, rechts von der Neuen Aula. Seid ihr auf der Hölderlinstraße, geht links dran vorbei und die Alte Physik ist auf der linken Straßenseite. Genaue Informationen bezüglich Treffpunkt am ersten Termin erhaltet ihr dann nochmal per Mail.
%
%\seticon{faBus}~\textbf{Bushaltestelle:} Gmelinstraße, Hölderlinstraße, Uni/Neue Aula
%
%\else
    \fi
\fi

\ifml
	\item~ % Funktioniert nicht anders, don't judge me
\else
    \item[Montag, 19. Oktober \YEAR]\ \\
  Heute beginnt der Vorbereitungskurs Mathematik. Es ist nicht Pflicht daran teilzunehmen,
	es ist aber sehr empfehlenswert.
	%Nicht zuletzt lernst du hier erste Freunde kennen!
	\ifsommersemester
	Der Vorkurs bietet dir eine Wiederholung des Schulstoffes sowie eine Übersicht über den Stoff von Mathe II
	\fi
	
	\ifwintersemester
	Der Vorkurs bietet dir eine Wiederholung des Schulstoffes sowie eine Übersicht über den Stoff von Mathe I
	\fi
	
	und führt dich in die Terminologie ein, die du in den Mathe-Vorlesungen wiederfinden wirst.
	
	\ifmaster
	\textbf{Wenn du deinen Bachelor nicht an der Universität Tübingen oder in einem anderen Fach erworben hast, kann der Vorkurs für dich sinnvoll sein. Auf unserer Website findest du ein Skript, von dem ein Teil auch im Vorkurs besprochen wird. Damit solltest du einschätzen können, wie viel vom Stoff bereits bekannt ist und ob sich der Besuch des Vorkurses lohnt.}
	\fi
	
    Um am Vorkurs teilzunehmen, musst du dich bis zum 14. Oktober \YEAR~anmelden. Weitere Informationen erhälst du nach Anmeldung zum Kurs per E-Mail. Falls der Brief aus unerfindlichen Gründen erst nach dem 14. Oktober bei dir eintrifft oder du sonstige Fragen dazu hast, melde dich bitte bei Rüdiger Zell \texttt{ruediger.zeller@uni-tuebingen.de}\\
    \url{https://uni-tuebingen.de/de/91877}.
	% Um Anmeldung wird gebeten, sie ist aber nicht zwingend erforderlich.

	\ifsommersemester
	%Der Vorkurs findet auf dem Gelände des Wilhelm-Schickard-Instituts (oft einfach nur \emph{der Sand} genannt) statt. Du erreichst den Sand per Bus mit den Linien 2 und 6, Haltestelle Sand Drosselweg. Dann ca. 200 Meter Richtung Süden und durch den großen gelben Torbogen. Per Auto erreichst du den Sand über den Nordring (folgt den Schildern "`Uni-Sand"'), hinter dem Hauptgebäude Sand 14 befindet sich ein großer Parkplatz. Der Hörsaal F119 befindet sich jedoch nicht im Hauptgebäude, sondern im Gebäude Sand 6. Nach dem Torbogen direkt links und an den Bäumen entlang, dann stehst du vor dem Haupteingang von Sand 6. Der Hörsaal befindet sich hinter der ersten Tür links.
	%Bitte beachtet, dass das Semesterticket \emph{erst ab dem 1.4} gültig ist. Falls du davor Tübingen schon unsicher machst und Bus fahren willst, musst du dir extra Fahrscheine kaufen.
	%Natürlich kommt man zur Morgenstelle auch zu Fuß oder mit dem Rad, da es jedoch steil den Berg hochgeht, muss man hierfür einiges an Zeit einrechnen.
	\fi

	%\seticon{faBus}~\textbf{Bushaltestelle:} Linie 2, Linie 6 "`Sand Drosselweg"' (Rest ausgeschildert)
\fi

\ifmaster
    \ifbinfo
        \item[Mittwoch, 21. Oktober \YEAR, TBA, Sand]\ \\
            Heute beginnt ein Informatik-Vorkurs speziell für Bioinformatik-Studenten im Master. Dieser Vorkurs wird dringend empfohlen, wenn du aus einem fachfremden Studiengang wie z.B. Biologie oder anderen Lebenswissenschaften kommst und noch keine oder sehr wenig Erfahrung in der Informatik und der Programmierung (CLI, Java, Python, \LaTeX) hast. Der Vorkurs wird in Englisch gehalten. Anmeldeschluss ist bis zum 20. Oktober, 12 Uhr. Alle weitere Informationen und die Anmeldung findest du auf folgender Website: \\ \url{https://uni-tuebingen.de/de/91881}

        \seticon{faBus}~\textbf{Bushaltestelle:} Linie 2, Linie 6 Sand Drosselweg
    \fi
\fi

\ifbachelor
    \iflehramt
        \item[Dienstag, 3. November \YEAR, TBA]\ \\
            Deine erste Vorlesung beginnt. \\
            Du hast um 14 Uhr „Informatik I“ bei \Infoprof.
            Alles, was du heute (und in Zukunft) benötigst: persönlichen Wachmacher, einen Stift, einen Block und den Studierendenausweis.

            %\seticon{faBus}~\textbf{Bushaltestelle:} BG Unfallklinik (Linie 5, 13, 14, 17, 18, 19, X15)
    \else
        \item[Montag, 2. November \YEAR, TBA]\ \\
            Deine erste Vorlesung beginnt. \\
            Du hast um 8 Uhr „Mathematik I“ bei \Matheprof.
            Alles, was du heute (und in Zukunft) benötigst: persönlichen Wachmacher, einen Stift, einen Block und den Studierendenausweis.

            %\seticon{faBus}~\textbf{Bushaltestelle:} BG Unfallklinik (Linie 5, 13, 14, 17, 18, 19, X15)
    \fi
\fi

\item[Veranstaltungen der Fachschaft]\ \\
	Natürlich planen wir auch fleißig für dich, wir werden diese hier eintragen, sobald wir alles fertig haben.

%Anfi-Mentorenprogramm
Dieses Semester bietet unser Fachbereich speziell ein Mentoren-Programm für Erstsemestler an. Dabei soll es ein regelmäßiges Treffen zwischen einem Professor und Kleingruppen an Studenten geben, in denen Fragen und Probleme geklärt werden. Weitere Details folgen.

\end{description}
